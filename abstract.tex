Identifying the ship's manoeuvring dynamics can be used to build models for ship manoeuvrability predictions which has a wide range of useful applications. A majority of the publications in this field are based on simulated data. In this paper, a new method for system identification is presented where the system is identified from model tests data.
The identification process can be decomposed into finding a suitable Vessel Manoeuvring Model (VMM) for the hydrodynamic forces and to correctly handle errors from the measurement noise. 
A Parameter Identification Technique (PIT) is proposed to identify  the VMM parameters (hydrodynamic derivatives) from measured model test trajectories and thrust. The most suitable VMM is found by using the PIT with cross validation on a set of competing VMMs.
The PIT uses inverse dynamics regression and Extended Kalman Filter (EKF) with a Rauch Tung Striebel (RTS) smoother. The multicollinearity problems in the VMMs are addressed by reducing the number of parameters and introducing the thrust force models. Two case study vessels: wPCC and KVLCC2 with very different maneuverability characteristics are used to demonstrate and validate the proposed method. 
Turning circle is used as the prediction case for both ships. Robust VMMs identified on model test data (excluding turning circle) show good agreement with the corresponding model test results for both ships. 
The use of EKF and RTS as a preprocessor to remove measurement noise and the use of cross validation to find a VMM with appropriate complexity produces VMMs with the capability of making predictions outside the training data.  
