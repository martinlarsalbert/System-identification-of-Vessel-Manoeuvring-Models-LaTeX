Identifying the ship's maneuvering dynamics can build models for ship maneuverability predictions
with a wide range of useful applications. A majority of the publications in this field are based on
simulated data. This paper presents a new system identification method where the system is
identified from model test data. The identification process can be decomposed into finding a suitable
Vessel Manoeuvring Model (VMM) for the hydrodynamic forces and to correctly handle errors from the
measurement noise. A Parameter Identification Technique (PIT) is proposed to identify the VMM parameter, 
such as hydrodynamic derivatives, from measured model test trajectories and thrust. The most suitable VMM
is found using the PIT with cross-validation on a set of competing VMMs. The PIT uses inverse
dynamics regression and Extended Kalman Filter (EKF) with a Rauch Tung Striebel (RTS) smoother. The
multicollinearity problems in the VMMs are addressed by reducing the number of parameters and introducing
the thrust force models. Two case study vessels, wPCC (Wind Powered Car Carrier) and KVLCC2 (Korean Very Large Crude Carrier), with very different maneuverability
characteristics are used to demonstrate and validate the proposed method. The turning circle is used as the
prediction case for both ships. Robust VMMs identified on model test data, excluding the turning circle, show
good agreement with the corresponding model test results for both ships. Using EKF and RTS as a
preprocessor to remove measurement noise and using cross-validation to find a VMM with appropriate
complexity produces VMMs with the capability of making predictions outside the training data.
