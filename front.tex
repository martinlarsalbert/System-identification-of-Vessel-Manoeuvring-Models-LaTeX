\begin{frontmatter}

  %% \title{\tnoteref{t1,t2}}
   %%\tnotetext[t1]{This document is a collaborative effort.}
   %%\tnotetext[t2]{The second title footnote which is longer 
   %%    than the first one and with an intention to fill
   %%   in up more than one line while formatting.} 
  
   %%\title\tnoteref{t1,t2}}
   %%\tnotetext[t1]{This document is the results of the research
   %%   project funded by the National Science Foundation.}
   %%\tnotetext[t2]{The second title footnote which is a longer 
   %%   text matter to fill through the whole text width and 
   %%   overflow into another line in the footnotes area of the 
   %%   first page.}
  
  \author[1,2]{Martin Alexandersson\corref{cor1}%
    \fnref{fn1}}
  \ead{maralex@chalmers.se}
  
  \author[1]{Wengang Mao\fnref{fn2}}
 %% \ead{wengang.mao@chalmers.se}
  
  \author[1]{Jonas W Ringsberg\fnref{fn2}}
  %%\ead{jonas.ringsberg@chalmers.se}
  
  \cortext[cor1]{Corresponding author}
  %%\fntext[fn1]{This is the first author footnote.}
  %%\fntext[fn2]{Another author footnote, this is a very long footnote and
  %% it should be a really long footnote. But this footnote is not yet
 %%   sufficiently long enough to make two lines of footnote text.}
 %% \fntext[fn3]{Yet another author footnote.}
  
  \affiliation[1]{organization={Dept. of Mechanics and Maritime Sciences, Division of Marine Technology,
                                Chalmers University of Technology},
                  addressline={Hörsalsvägen 7A}, 
                  city={Gothenburg},
  %               citysep={}, % Uncomment if no comma needed between city and postcode
                  postcode={41296}, 
                  state={Gothenburg},
                  country={Sweden}}
  
  \affiliation[2]{organization={SSPA Sweden AB},
                  addressline={Chalmers tvärgata 10}, 
                  postcode={41296}, 
                  postcodesep={}, 
                  city={Gothenburg,},
                  country={Sweden}}
  
  
  \begin{abstract}
  Identifying the ship's maneuvering dynamics can build models for ship maneuverability predictions with a wide range of useful applications. 
A majority of the publications in this field are based on simulated data. In this paper model test data is used. The identification process can be decomposed into finding a suitable Vessel Manoeuvring Model (VMM) for the hydrodynamic forces and to correctly handle errors from the measurement noise. A Parameter Identification Technique (PIT) is proposed to identify the hydrodynamic derivatives. The most suitable VMM is found using the PIT with cross-validation on a set of competing VMMs. The PIT uses inverse dynamics regression and Extended Kalman Filter (EKF) with a Rauch Tung Striebel (RTS) smoother. Two case study vessels, wPCC and KVLCC2, with very different maneuverability characteristics are used to demonstrate and validate the proposed method. Turning circle predictions with the Robust VMMs, trained on zigzag model tests, show good agreement with the corresponding model test results for both ships.

  \end{abstract}
  
  \begin{keyword}
    Ship Manoeuvring, System Identification, Inverse Dynamics, Extended Kalman Filter, RTS smoother, Multicollinearity
  \end{keyword}
  
  \end{frontmatter}