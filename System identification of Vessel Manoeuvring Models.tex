%% Generated by Sphinx.
\def\sphinxdocclass{jupyterBook}
\documentclass[review]{elsarticle}
\ifdefined\pdfpxdimen
   \let\sphinxpxdimen\pdfpxdimen\else\newdimen\sphinxpxdimen
\fi \sphinxpxdimen=.75bp\relax
\ifdefined\pdfimageresolution
    \pdfimageresolution= \numexpr \dimexpr1in\relax/\sphinxpxdimen\relax
\fi
%% let collapsible pdf bookmarks panel have high depth per default
\PassOptionsToPackage{bookmarksdepth=5}{hyperref}
%% turn off hyperref patch of \index as sphinx.xdy xindy module takes care of
%% suitable \hyperpage mark-up, working around hyperref-xindy incompatibility
\PassOptionsToPackage{hyperindex=false}{hyperref}
%% memoir class requires extra handling
\makeatletter\@ifclassloaded{memoir}
{\ifdefined\memhyperindexfalse\memhyperindexfalse\fi}{}\makeatother

\PassOptionsToPackage{warn}{textcomp}


\usepackage{cmap}

\usepackage{amsmath,amssymb,amstext}
\usepackage{polyglossia}
\setmainlanguage{english}






\usepackage[,numfigreset=1,mathnumfig]{sphinx}

\fvset{fontsize=\small}
\usepackage{geometry}


% Include hyperref last.

\usepackage{hyperref}
\def\equationautorefname~#1\null{Eq.#1\null}
% Fix anchor placement for figures with captions.
\usepackage{hypcap}% it must be loaded after hyperref.
% Set up styles of URL: it should be placed after hyperref.
\urlstyle{same}


\usepackage{sphinxmessages}



        % Start of preamble defined in sphinx-jupyterbook-latex %
         \usepackage[Latin,Greek]{ucharclasses}
        \usepackage{unicode-math}
        % fixing title of the toc
        \addto\captionsenglish{\renewcommand{\contentsname}{Contents}}
        \hypersetup{
            pdfencoding=auto,
            psdextra
        }
        % End of preamble defined in sphinx-jupyterbook-latex %
        

\title{System identification of Vessel Manoeuvring Models}
\date{Oct 04, 2022}
\release{}

\newcommand{\sphinxlogo}{\vbox{}}
\renewcommand{\releasename}{}
\makeindex
\bibliographystyle{elsarticle-num}
\begin{document}
\input{front}


\pagestyle{empty}

\pagestyle{plain}

\pagestyle{normal}
\phantomsection\label{\detokenize{index::doc}}



\section{Introduction}
\label{\detokenize{00.02_introduction:introduction}}\label{\detokenize{00.02_introduction::doc}}
\sphinxAtStartPar
Ship manoeuvring performance can be assessed with varying accuracy, effort, and cost. The full\sphinxhyphen{}scale manoeuvring test during sea trials is the most common method used to demonstrate compliance with the International Maritime Organization (IMO) manoeuvring criteria \cite{imo_standards_2002}, which all ships over 100 meters must fulfill.
However before ships are built, manoeuvring prediction methods are needed. The free model test \cite{ittc_ittc_2008} is often recognized as the most accurate prediction method \cite{ittc_maneuvering_2008}, which can also be conducted with high accuracy in CFD \cite{araki_estimating_2012}. These methods are expensive and they also have the drawback of obtaining only results for one specific maneuver at the time. Some complex maneuvers, such as harbor maneuvers, are also very hard to conduct in this way.
The system\sphinxhyphen{}based manoeuvring model is a more cost\sphinxhyphen{}efficient solution with many applications such as the maneuverability requirement for ship design and construction, the design of advanced ship autopilot systems, unmanned surface vehicles (USVs) \cite{bai_review_2022} or master mariners’ training simulators.\\
Captive model tests \cite{ittc_ittc_2008-1}, Planar Motion Mechanism tests (PMM) \cite{ittc_ittc_2008-1} or Virtual captive tests (VCT) \cite{liu_predictions_2018} can be conducted as input to the manoeuvring model. These tests are also costly, and the increased flexibility of the mathematical model gives lower accuracy \cite{ittc_maneuvering_2008}.

\sphinxAtStartPar
System identification methods applied on recorded ship trajectories from free model tests, CFD free running trials \cite{araki_estimating_2012} or full scale ship operation offers a more cost\sphinxhyphen{}efficient way to develop system\sphinxhyphen{}based manoeuvring models.
Some of the publications within this field are summarized in \hyperref[\detokenize{00.02_introduction:tab-methods}]{Table \ref{\detokenize{00.02_introduction:tab-methods}}}. The methods identifies models that can be categorized into three groups: white\sphinxhyphen{}box (WB), grey\sphinxhyphen{}box (GB) or black\sphinxhyphen{}box models (BB) \cite{miller_ship_2021}. The white\sphinxhyphen{}box models are deterministic and are not data driven by applying only physical principles, for instance with semi\sphinxhyphen{}empirical formulas which have very low cost but also very low accuracy \cite{ittc_maneuvering_2008}. The black\sphinxhyphen{}box models are stochastic and data driven. No knowledge about the system structure and parameters is required \cite{miller_ship_2021}. The grey\sphinxhyphen{}box modeling is a combination of white\sphinxhyphen{}box and black\sphinxhyphen{}box modeling methods, so that both a physical model and data is used.

\sphinxAtStartPar
The system identification can be applied on full scale data \cite{astrom_identification_1976}, \cite{perera_system_2015}, \cite{revestido_herrero_two-step_2012} which has the highest uncertainty, both in terms of model uncertainty and measurement uncertainty which is therefore the hardest task, but also the most relevant. The uncertainty can be reduced by instead using model test data as in \cite{araki_estimating_2012}, \cite{he_nonparametric_2022}, \cite{xue_identification_2021}, \cite{miller_ship_2021} and \cite{luo_parameter_2016} . The uncertainty can be further reduced by using simulated data as in \cite{shi_identification_2009}, \cite{zhu_parameter_2017}, \cite{wang_parameter_2021} which can show the potential of new methods with the benefit that the true model is known, but one also has to remember that the objective is to identify real objects, not its mathematical model \cite{miller_ship_2021}.


\begin{savenotes}\sphinxattablestart
\centering
\sphinxcapstartof{table}
\sphinxthecaptionisattop
\sphinxcaption{System identification references}\label{\detokenize{00.02_introduction:tab-methods}}
\sphinxaftertopcaption
\begin{tabulary}{\linewidth}[t]{|T|T|T|T|T|}
\hline
\sphinxstyletheadfamily 
\sphinxAtStartPar
Method
&\sphinxstyletheadfamily 
\sphinxAtStartPar
BB
&\sphinxstyletheadfamily 
\sphinxAtStartPar
GB
&\sphinxstyletheadfamily 
\sphinxAtStartPar
Data
&\sphinxstyletheadfamily 
\sphinxAtStartPar
Reference
\\
\hline
\sphinxAtStartPar
Constrained Least Squares
&
\sphinxAtStartPar
x
&&
\sphinxAtStartPar
model test, CFD
&
\sphinxAtStartPar
\cite{araki_estimating_2012}
\\
\hline
\sphinxAtStartPar
Neural network
&
\sphinxAtStartPar
x
&&
\sphinxAtStartPar
model test
&
\sphinxAtStartPar
\cite{he_nonparametric_2022}
\\
\hline
\sphinxAtStartPar
Gaussian process
&
\sphinxAtStartPar
x
&&
\sphinxAtStartPar
model test
&
\sphinxAtStartPar
\cite{xue_identification_2021}
\\
\hline
\sphinxAtStartPar
Kalman filter maximum likelihood
&&
\sphinxAtStartPar
x
&
\sphinxAtStartPar
full scale
&
\sphinxAtStartPar
\cite{astrom_identification_1976}
\\
\hline
\sphinxAtStartPar
Unscented kalman filter
&&
\sphinxAtStartPar
x
&
\sphinxAtStartPar
full scale
&
\sphinxAtStartPar
\cite{revestido_herrero_two-step_2012}
\\
\hline
\sphinxAtStartPar
Extended kalman filter
&&
\sphinxAtStartPar
x
&
\sphinxAtStartPar
full scale
&
\sphinxAtStartPar
\cite{perera_system_2015}
\\
\hline
\sphinxAtStartPar
Extended kalman filter
&&
\sphinxAtStartPar
x
&
\sphinxAtStartPar
simulated
&
\sphinxAtStartPar
\cite{shi_identification_2009}
\\
\hline
\sphinxAtStartPar
SVR
&&
\sphinxAtStartPar
x
&
\sphinxAtStartPar
simulated
&
\sphinxAtStartPar
\cite{zhu_parameter_2017}, \cite{wang_parameter_2021}
\\
\hline
\sphinxAtStartPar
SVR
&&
\sphinxAtStartPar
x
&
\sphinxAtStartPar
model test
&
\sphinxAtStartPar
\cite{luo_parameter_2016}
\\
\hline
\sphinxAtStartPar
Genetic algorithm
&&
\sphinxAtStartPar
x
&
\sphinxAtStartPar
lake test
&
\sphinxAtStartPar
\cite{miller_ship_2021}
\\
\hline
\end{tabulary}
\par
\sphinxattableend\end{savenotes}

\sphinxAtStartPar
Black\sphinxhyphen{}box modeling was used in \cite{he_nonparametric_2022} using neural network and in \cite{xue_identification_2021} using gaussian process. The nonparametric models are related, where the system structure is known but no parameters are required as seen in \cite{pongduang_nonparametric_2020}. However, most of the system identification methods for ship manoeuvring models use the grey\sphinxhyphen{}box modeling by assuming a predefined mathematical model, which reduces the problem to a parameter estimation.
The Kalman Filter (KF) combined with Maximum Likelihood Estimation was proposed already in 1976 \cite{astrom_identification_1976} to develop a linear manoeuvring model based on manually recorded data in 1969 onboard the Atlantic Song freighter. The Extended Kalman Filter (EKF) can also estimate parameters if the parameters are represented as states of the state space model. This technique was used on a nonlinear Nomoto model \cite{perera_system_2015} and a 3 degree of freedom model (3DOF) \cite{shi_identification_2009}. EKF was used in \cite{araki_estimating_2012} with constrained parameters based on physical reasoning and prior knowledge using constrained least squares regression. Unscented Kalman Filter (UKF), which has been proposed as an improvement to the EKF in handling nonlinear systems, was used in \cite{revestido_herrero_two-step_2012}.
Support Vector Regression (SVR) has been investigated in \cite{zhu_parameter_2017}, \cite{wang_parameter_2021} and \cite{luo_parameter_2016}. A genetic algorithm was used in \cite{miller_ship_2021} for the system identification of model test performed on a lake.

\sphinxAtStartPar
The drift effect of the hydrodynamic coefficients is inevitable in modeling of ship manoeuvring motions by using the system identification. The drifts of hydrodynamic coefficients result from the so\sphinxhyphen{}called multicollinearity, meaning that if the input variables of a regression model are firmly linearly dependent on each other, the regression results of their coefficients may be incorrect \cite{luo_parameter_2016}. The identified coefficients within the mathematical model do not have to be physically correct but mathematically correct \cite{ittc_maneuvering_2008}. Many nonlinear hydrodynamic coefficients in the mathematical model of ship manoeuvring motions have no apparent physical meaning \cite{luo_parameter_2016}. Multicollinearity can be reduced by reducing the number of parameters in the model, either by simplification or adding more deterministic parts by including hydrodynamic relations. For instance, slender body theory can be added \cite{hwang_cancellation_1982}. Simplification can be based on hydrodynamic reasoning where the number of parameters can be reduced \cite{luo_parameter_2016}.

\sphinxAtStartPar
However, system identification methods must handle imperfections in the data from measurement noise and model uncertainty of the manoeuvring model, which will always be present since the model can never perfectly capture all fundamental physics. When developing a manoeuvring model with model test data for manoeuvring prediction, preprocessing of data and a method to choose an appropriate manoeuvring model is needed \cite{alexandersson_comparison_2022}.
This paper proposes an innovative approach to address those issues for ship manoeuvring system identification based on actual noise test data. First, a parameter estimation method  is proposed to study the capability of several candidate manoeuvring models. Model test data in all degrees of freedom is recorded as ship trajectories, including position and heading and thrust, where a model for propeller thrust is also proposed and further developed. In this method, the inverse dynamics regression and Extended the Kalman Filter (EKF) with a Rauch Tung Striebel (RTS) smoother \cite{rauch_maximum_1965} are proposed for the necessary data cleaning before feeding into the manoeuvring model. An iterative approach is adopted to determine the most appropriate manoeuvring model. The multicollinearity problem is addressed by identifying a ship manoeuvring system with proper parameters in the manoeuvring model. The proposed approach is verified by identifying a selected manoeuvring model and predicting turning circle manoeuvres for two different test vessels.

\sphinxAtStartPar
For the completeness of this paper, different manoeuvring models and propeller models are briefly introduced in \hyperref[\detokenize{02.01_VMMs:vmm}]{Section \ref{\detokenize{02.01_VMMs:vmm}}}. Then the proposed parameter estimation method is presented in \hyperref[\detokenize{01.01_method:method}]{Section \ref{\detokenize{01.01_method:method}}}, where each subcomponent is also described. The parameter estimation is applied to two case study ships which are introduced in \hyperref[\detokenize{05.01_case_studies:case-studies}]{Section \ref{\detokenize{05.01_case_studies:case-studies}}} and corresponding results are presented in the \hyperref[\detokenize{06.01_results:results}]{Section \ref{\detokenize{06.01_results:results}}} with discussions, followed by conclusions.


\section{Vessel Manoeuvring Models}
\label{\detokenize{02.01_VMMs:vessel-manoeuvring-models}}\label{\detokenize{02.01_VMMs:vmm}}\label{\detokenize{02.01_VMMs::doc}}
\sphinxAtStartPar
Ship manoeuvring is a simplified case of seakeeping. The encountering waves have been removed, assuming calm water conditions. This simplification allows the ship dynamics to be expressed with only four degrees of freedom: surge, sway, roll, and yaw, where the roll is often excluded. Surge, sway, and yaw have very low frequencies during manoeuvres, so added masses and other hydrodynamic derivatives can be assumed as constants  \cite{fossen_handbook_2021}. Three manoeuvring models are used in this paper: the Linear (LVMM) \cite{matusiak_dynamics_2017}, the Abkowitz (AVMM), \cite{abkowitz_ship_1964} and the Modified Abkowitz (MAVMM), proposed in this paper.
\hyperref[\detokenize{02.01_VMMs:coordinate-system}]{Fig.\@ \ref{\detokenize{02.01_VMMs:coordinate-system}}} shows the reference frames used in the manoeuvring models where \(x_0\) and \(y_0\) and heading \(\Psi\) are the global position and orientation of a ship fix reference frame \(O(x,y,z)\) (or rather \(O(x,y)\) when heave is exluded) with origin at midship. \(u\), \(v\), \(r\), \(X\), \(Y\) and \(N\) are velocities and forces in the ship fix reference frame.

\begin{figure}[H]
\centering
\capstart

\noindent\sphinxincludegraphics[height=300\sphinxpxdimen]{{coordinate_system}.PNG}
\caption{Coordinate system}\label{\detokenize{02.01_VMMs:coordinate-system}}\end{figure}

\sphinxAtStartPar
The acceleration can be solved from the manoeuvring equation (\autoref{equation:02.01_VMMs:eqqsystem}) \cite{fossen_handbook_2021} as seen in \autoref{equation:02.01_VMMs:eqacc},
\begin{equation}\label{equation:02.01_VMMs:eqqsystem}
\begin{split}\displaystyle \left[\begin{matrix}- X_{\dot{u}} + m & 0 & 0\\0 & - Y_{\dot{v}} + m & - Y_{\dot{r}} + m x_{G}\\0 & - N_{\dot{v}} + m x_{G} & I_{z} - N_{\dot{r}}\end{matrix}\right] \left[\begin{matrix}\dot{u}\\\dot{v}\\\dot{r}\end{matrix}\right] = \left[\begin{matrix}m r^{2} x_{G} + m r v + \operatorname{X_{D}}{\left(u,v,r,\delta,thrust \right)}\\- m r u + \operatorname{Y_{D}}{\left(u,v,r,\delta,thrust \right)}\\- m r u x_{G} + \operatorname{N_{D}}{\left(u,v,r,\delta,thrust \right)}\end{matrix}\right]\end{split}
\end{equation}\begin{equation}\label{equation:02.01_VMMs:eqacc}
\begin{split}\displaystyle \dot{\nu} = \left[\begin{matrix}\dot{u}\\\dot{v}\\\dot{r}\end{matrix}\right] = \left[\begin{matrix}\frac{1}{- X_{\dot{u}} + m} & 0 & 0\\0 & - \frac{- I_{z} + N_{\dot{r}}}{S} & - \frac{- Y_{\dot{r}} + m x_{G}}{S}\\0 & - \frac{- N_{\dot{v}} + m x_{G}}{S} & - \frac{Y_{\dot{v}} - m}{S}\end{matrix}\right] \left[\begin{matrix}m r^{2} x_{G} + m r v + \operatorname{X_{D}}{\left(u,v,r,\delta,thrust \right)}\\- m r u + \operatorname{Y_{D}}{\left(u,v,r,\delta,thrust \right)}\\- m r u x_{G} + \operatorname{N_{D}}{\left(u,v,r,\delta,thrust \right)}\end{matrix}\right]\end{split}
\end{equation}
\sphinxAtStartPar
where \(S\) is a helper variable:
\begin{equation}\label{equation:02.01_VMMs:eq_S}
\begin{split}\displaystyle S = - I_{z} Y_{\dot{v}} + I_{z} m + N_{\dot{r}} Y_{\dot{v}} - N_{\dot{r}} m - N_{\dot{v}} Y_{\dot{r}} + N_{\dot{v}} m x_{G} + Y_{\dot{r}} m x_{G} - m^{2} x_{G}^{2}\end{split}
\end{equation}
\sphinxAtStartPar
A state space model for manoeuvring can now be defined with six states:
\begin{equation}\label{equation:02.01_VMMs:eq_x}
\begin{split}\displaystyle \mathbf{x} = \left[\begin{matrix}x_{0}\\y_{0}\\\Psi\\u\\v\\r\end{matrix}\right]\end{split}
\end{equation}
\sphinxAtStartPar
The time derivative of this state \(\dot{\mathbf{x}}\) can be defined by a state transition \(f(\mathbf{x},\mathbf{c})\) using geometrical relations
how global coordinates \(x_0\), \(y_0\) and \(\Psi\) depend on \(u\), \(v\), and \(r\) viz.,
\begin{equation}\label{equation:02.01_VMMs:eqf}
\begin{split}\displaystyle \dot{\mathbf{x}} = f(\mathbf{x},\mathbf{c}) + \mathbf{w}
                                          = \left[\begin{matrix}\dot{x_0}\\ \dot{y_0} \\ \dot{\Psi} \\\dot{u}\\\dot{v}\\\dot{r}\end{matrix}\right] + \mathbf{w}
                                          = \left[\begin{matrix}u \cos{\left(\Psi \right)} - v \sin{\left(\Psi \right)}\\u \sin{\left(\Psi \right)} + v \cos{\left(\Psi \right)}\\r\\\dot{u}\\\dot{v}\\\dot{r}\end{matrix}\right] + \mathbf{w}\end{split}
\end{equation}
\sphinxAtStartPar
where \(\mathbf{c}\) is control inputs (rudder angle \(\delta\) and thrust); the last three derivatives: \(\dot{u}\), \(\dot{v}\), \(\dot{r}\) are calculated with \autoref{equation:02.01_VMMs:eqacc}.
\(\mathbf{w}\) is the process noise, i.e., the difference between the predicted state by the manoeuvring model and the true
state of the system. \(\mathbf{w}\) is unknown when the manoeuvring model is used for manoeuvre predictions and therefore normally
assumed to be zero, but it is an important factor when the manoeuvring model is used in the EKF, see Section \(\ref{\detokenize{04.01_EK::doc}}\).
The manoeuvring simulation can now be conducted by numerical integration of \autoref{equation:02.01_VMMs:eqf}. The main difference between the manoeuvring models lies in how the hydrodynamic functions \(X_D(u,v,r,\delta,thrust)\), \(Y_D(u,v,r,\delta,thrust)\), \(N_D(u,v,r,\delta,thrust)\) are defined. These expressions are denoted below for the various manoeuvring models: LVMM, AVMM and MAVMM.

\sphinxAtStartPar
LVMM (Linear Vessel Manoeuvring Model) \cite{matusiak_dynamics_2017}:
\begin{equation}\label{equation:02.01_VMMs:eqxlinear}
\begin{split}\begin{split}
\operatorname{X_{D}'}{\left(u',v',r',\delta,thrust' \right)} = & X_{\delta} \delta + X_{r} r' + X_{u} u' + X_{v} v' 
\end{split}\end{split}
\end{equation}\begin{equation}\label{equation:02.01_VMMs:eqylinear}
\begin{split}\begin{split}
\operatorname{Y_{D}'}{\left(u',v',r',\delta,thrust' \right)} = & Y_{\delta} \delta + Y_{r} r' + Y_{u} u' + Y_{v} v' 
\end{split}\end{split}
\end{equation}\begin{equation}\label{equation:02.01_VMMs:eqnlinear}
\begin{split}\begin{split}
\operatorname{N_{D}'}{\left(u',v',r',\delta,thrust' \right)} = & N_{\delta} \delta + N_{r} r' + N_{u} u' + N_{v} v' 
\end{split}\end{split}
\end{equation}
\sphinxAtStartPar
AVMM (Abkowitz Vessel Manoeuvring Model) \cite{abkowitz_ship_1964}:
\begin{equation}\label{equation:02.01_VMMs:eqxabkowitz}
\begin{split}\begin{split}
\operatorname{X_{D}'}{\left(u',v',r',\delta,thrust' \right)} = & X_{\delta\delta} \delta^{2} + X_{r\delta} \delta r' + X_{rr} r'^{2} + X_{T} thrust' + X_{u\delta\delta} \delta^{2} u' + X_{ur\delta} \delta r' u' + X_{urr} r'^{2} u' + X_{uuu} u'^{3} \\
& + X_{uu} u'^{2} + X_{uv\delta} \delta u' v' + X_{uvr} r' u' v' + X_{uvv} u' v'^{2} + X_{u} u' + X_{v\delta} \delta v' + X_{vr} r' v' + X_{vv} v'^{2} 
\end{split}\end{split}
\end{equation}\begin{equation}\label{equation:02.01_VMMs:eqyabkowitz}
\begin{split}\begin{split}
\operatorname{Y_{D}'}{\left(u',v',r',\delta,thrust' \right)} = & Y_{0uu} u'^{2} + Y_{0u} u' + Y_{0} + Y_{\delta\delta\delta} \delta^{3} + Y_{\delta} \delta + Y_{r\delta\delta} \delta^{2} r' + Y_{rr\delta} \delta r'^{2} + Y_{rrr} r'^{3} \\
& + Y_{r} r' + Y_{T\delta} \delta thrust' + Y_{T} thrust' + Y_{u\delta} \delta u' + Y_{ur} r' u' + Y_{uu\delta} \delta u'^{2} + Y_{uur} r' u'^{2} + Y_{uuv} u'^{2} v' \\
& + Y_{uv} u' v' + Y_{v\delta\delta} \delta^{2} v' + Y_{vr\delta} \delta r' v' + Y_{vrr} r'^{2} v' + Y_{vv\delta} \delta v'^{2} + Y_{vvr} r' v'^{2} + Y_{vvv} v'^{3} + Y_{v} v' 
\end{split}\end{split}
\end{equation}\begin{equation}\label{equation:02.01_VMMs:eqnabkowitz}
\begin{split}\begin{split}
\operatorname{N_{D}'}{\left(u',v',r',\delta,thrust' \right)} = & N_{0uu} u'^{2} + N_{0u} u' + N_{0} + N_{\delta\delta\delta} \delta^{3} + N_{\delta} \delta + N_{r\delta\delta} \delta^{2} r' + N_{rr\delta} \delta r'^{2} + N_{rrr} r'^{3} \\
& + N_{r} r' + N_{T\delta} \delta thrust' + N_{T} thrust' + N_{u\delta} \delta u' + N_{ur} r' u' + N_{uu\delta} \delta u'^{2} + N_{uur} r' u'^{2} + N_{uuv} u'^{2} v' \\
& + N_{uv} u' v' + N_{v\delta\delta} \delta^{2} v' + N_{vr\delta} \delta r' v' + N_{vrr} r'^{2} v' + N_{vv\delta} \delta v'^{2} + N_{vvr} r' v'^{2} + N_{vvv} v'^{3} + N_{v} v' 
\end{split}\end{split}
\end{equation}
\sphinxAtStartPar
MAVMM (Modified Abkowitz Vessel Manoeuvring Model, where only the most relevant coefficients in AVMM are included.)
\begin{equation}\label{equation:02.01_VMMs:eqxmartinssimple}
\begin{split}\begin{split}
\operatorname{X_{D}'}{\left(u',v',r',\delta,thrust' \right)} = & X_{\delta\delta} \delta^{2} + X_{rr} r'^{2} + X_{T} thrust' + X_{uu} u'^{2} + X_{u} u' + X_{vr} r' v' 
\end{split}\end{split}
\end{equation}\begin{equation}\label{equation:02.01_VMMs:eqymartinssimple}
\begin{split}\begin{split}
\operatorname{Y_{D}'}{\left(u',v',r',\delta,thrust' \right)} = & Y_{\delta} \delta + Y_{r} r' + Y_{T\delta} \delta thrust' + Y_{T} thrust' + Y_{ur} r' u' + Y_{u} u' + Y_{vv\delta} \delta v'^{2} + Y_{v} v' 
\end{split}\end{split}
\end{equation}\begin{equation}\label{equation:02.01_VMMs:eqnmartinssimple}
\begin{split}\begin{split}
\operatorname{N_{D}'}{\left(u',v',r',\delta,thrust' \right)} = & N_{\delta} \delta + N_{r} r' + N_{T\delta} \delta thrust' + N_{T} thrust' + N_{ur} r' u' + N_{u} u' + N_{vv\delta} \delta v'^{2} + N_{v} v' 
\end{split}\end{split}
\end{equation}
\sphinxAtStartPar
The hydrodynamic functions above are expressed using nondimensional units with the prime system, denoted by the prime symbol (\('\)). The quantities are expressed in the prime system, using the denominators in \hyperref[\detokenize{02.01_VMMs:prime-system-denominators}]{Table \ref{\detokenize{02.01_VMMs:prime-system-denominators}}}. For instance, surge linear velocity \(u\) can be expressed in the prime system as seen in \autoref{equation:02.01_VMMs:eqprime} using the linear velocity denominator.
\begin{equation}\label{equation:02.01_VMMs:eqprime}
\begin{split}\displaystyle u'=\frac{u}{V}\end{split}
\end{equation}
\sphinxAtStartPar
Equations can either be written in the prime or regular SI system. The hydrodynamic derivatives are always expressing forces in the prime system as function of state variables. The (\('\)) sign is therefore implicit and not written out as seen in \autoref{equation:02.01_VMMs:eqderivativeprime}.
\begin{equation}\label{equation:02.01_VMMs:eqderivativeprime}
\begin{split}\displaystyle Y_{\delta'}'=\frac{\partial Y_D'}{\partial \delta'} := Y_{\delta} \end{split}
\end{equation}
\sphinxAtStartPar
The exceptions are the added masses (\(X_{\dot{u}}\), \(Y_{\dot{v}}\), \(Y_{\dot{r}}\), \(N_{\dot{v}}\) and \(N_{\dot{r}}\)) which are expressed in both Prime system or the regular SI system where the (\('\)) sign is therefore
explicitly stated.
There is however a great benefit in expressing the hydrodynamic forces in the prime system. The forces are often nonlinear due to a quadratic relation to the flow velocity, as seen in \autoref{equation:02.01_VMMs:eqquadraticsi}.
\begin{equation}\label{equation:02.01_VMMs:eqquadraticsi}
\begin{split}\displaystyle Y_{D}=Y_{\delta} \cdot \delta \cdot \frac{L^2V^2\rho}{2}\end{split}
\end{equation}
\sphinxAtStartPar
which becomes linear when expressed in the prime system as seen in \autoref{equation:02.01_VMMs:eqquadraticprime}.
\begin{equation}\label{equation:02.01_VMMs:eqquadraticprime}
\begin{split}\displaystyle Y_{D}'=Y_{\delta} \cdot \delta'\end{split}
\end{equation}

\begin{savenotes}\sphinxattablestart
\centering
\sphinxcapstartof{table}
\sphinxthecaptionisattop
\sphinxcaption{Prime system denominators}\label{\detokenize{02.01_VMMs:prime-system-denominators}}
\sphinxaftertopcaption
\begin{tabulary}{\linewidth}[t]{|T|T|}
\hline
\sphinxstyletheadfamily &\sphinxstyletheadfamily 
\sphinxAtStartPar
Denominators
\\
\hline
\sphinxAtStartPar
angle
&
\sphinxAtStartPar
\(1\)
\\
\hline
\sphinxAtStartPar
angular
acceleration
&
\sphinxAtStartPar
\(\frac{V^{2}}{L^{2}}\)
\\
\hline
\sphinxAtStartPar
angular
velocity
&
\sphinxAtStartPar
\(\frac{V}{L}\)
\\
\hline
\sphinxAtStartPar
area
&
\sphinxAtStartPar
\(L^{2}\)
\\
\hline
\sphinxAtStartPar
density
&
\sphinxAtStartPar
\(\frac{\rho}{2}\)
\\
\hline
\sphinxAtStartPar
force
&
\sphinxAtStartPar
\(\frac{L^{2} V^{2} \rho}{2}\)
\\
\hline
\sphinxAtStartPar
frequency
&
\sphinxAtStartPar
\(\frac{V}{L}\)
\\
\hline
\sphinxAtStartPar
inertia
moment
&
\sphinxAtStartPar
\(\frac{L^{5} \rho}{2}\)
\\
\hline
\sphinxAtStartPar
length
&
\sphinxAtStartPar
\(L\)
\\
\hline
\sphinxAtStartPar
linear
acceleration
&
\sphinxAtStartPar
\(\frac{V^{2}}{L}\)
\\
\hline
\sphinxAtStartPar
linear
velocity
&
\sphinxAtStartPar
\(V\)
\\
\hline
\sphinxAtStartPar
mass
&
\sphinxAtStartPar
\(\frac{L^{3} \rho}{2}\)
\\
\hline
\sphinxAtStartPar
moment
&
\sphinxAtStartPar
\(\frac{L^{3} V^{2} \rho}{2}\)
\\
\hline
\sphinxAtStartPar
time
&
\sphinxAtStartPar
\(\frac{L}{V}\)
\\
\hline
\sphinxAtStartPar
volume
&
\sphinxAtStartPar
\(L^{3}\)
\\
\hline
\end{tabulary}
\par
\sphinxattableend\end{savenotes}


\subsection{The propeller model}
\label{\detokenize{02.10_propeller_model:the-propeller-model}}\label{\detokenize{02.10_propeller_model::doc}}
\sphinxAtStartPar
The propeller model is developed based on Manoeuvring Modeling Group (MMG) model \cite{yasukawa_introduction_2015-1} where the thrust is expressed as:
\begin{equation}\label{equation:02.10_propeller_model:eqT}
\begin{split}\displaystyle thrust = D^{4} K_{T} n^{2} \rho\end{split}
\end{equation}
\sphinxAtStartPar
and the thrust coefficient \(K_T\) is modelled as a second order polynomial:
\begin{equation}\label{equation:02.10_propeller_model:eqkt}
\begin{split}\displaystyle K_{T} = J^{2} k_{2} + J k_{1} + k_{0}\end{split}
\end{equation}
\sphinxAtStartPar
The advance ration \(J\) is calculated as:
\begin{equation}\label{equation:02.10_propeller_model:eqJ}
\begin{split}\displaystyle J = \frac{u \left(1 - w_{p}\right)}{D n}\end{split}
\end{equation}
\sphinxAtStartPar
where \(D\) is propeller diameter, \(n\) is propeller speed and \(w_p\) is the wake fraction at an oblique inflow to the propeller from the drift angle and the yaw rate. A semi\sphinxhyphen{}empirical formula for \(w_p\) is provided in the MMG model. As an alternative, a simple polynomial is proposed in \autoref{equation:02.10_propeller_model:eqpropellermodel}.
\begin{equation}\label{equation:02.10_propeller_model:eqpropellermodel}
\begin{split}\displaystyle w_{p} = C_{1} \delta + C_{2} \delta^{2} + C_{3} \beta_{p}^{2} + C_{4} u + w_{p0}\end{split}
\end{equation}
\sphinxAtStartPar
\(w_p\) is modeled as a function of rudder angle \(\delta\), to include wake influence from the rudder and ship speed \(u\), to include a speed dependency. The influence from drift angle \(\beta\) and yaw rate \(r\) is expressed by \(\beta_p\) in \autoref{equation:02.10_propeller_model:eqbetap}.
\begin{equation}\label{equation:02.10_propeller_model:eqbetap}
\begin{split}\beta_p=\beta - \frac{r}{V} \cdot x_p \end{split}
\end{equation}
\sphinxAtStartPar
Where \(x_p\) is the propeller longitudinal position and \(w_{p0}\) is the regular Taylor wake fraction, applicable to straight ahead steaming with no rudder angle. Similar to the MMG propeller model, two sets of parameters \(C_1\)\sphinxhyphen{}\(C_4\) should be used in the propeller model depending on the sign of \(\beta_p\).


\section{Method}
\label{\detokenize{01.01_method:method}}\label{\detokenize{01.01_method:id1}}\label{\detokenize{01.01_method::doc}}
\sphinxAtStartPar
An efficient approach to build the manoeuvring model for a ship’s manoeuvres is presented in this paper. In this procedure, an initial manoeuvring model is used to solve the reversed manoeuvring problem, i.e., predicting unknown forces from known ship manoeuvres. Then, the hydrodynamic derivatives in the manoeuvring model can be identified with regression of the force polynomials on forces predicted with inverse dynamics, which is called motion regression in this paper. The Ordinary Least Square (OLS) method is used to regress the hydrodynamic derivatives in the proposed parameter estimation method. The OLS is known to be extremely sensitive to noise and outliers inevitably associated with both experimental and full\sphinxhyphen{}scale test data. Thereby, the focus in the present parameter estimation is on pre\sphinxhyphen{}processing data with filtering rather than the regression method itself. Both the Extended Kalman Filter (EKF) and Rauch Tung Striebel (RTS) smoother are used to perform the data\sphinxhyphen{}processing for building a proper manoeuvring model.

\begin{figure}[H]
\centering
\capstart

\noindent\sphinxincludegraphics[width=500\sphinxpxdimen]{{method}.png}
\caption{Flow chart over the proposed parameter estimation method}\label{\detokenize{01.01_method:overview}}\end{figure}


\subsection{Overview of the proposed procedure}
\label{\detokenize{01.01_method:overview-of-the-proposed-procedure}}
\sphinxAtStartPar
After choosing a proper manoeuvring model to describe a ship’s manoeuvring performance, the coefficients in those manoeuvring models can be estimated by the proposed parameter estimation method composed of two basic steps similar to \cite{revestido_herrero_two-step_2012} as shown in \hyperref[\detokenize{01.01_method:overview}]{Fig.\@ \ref{\detokenize{01.01_method:overview}}}.
The measurement noise needs to be removed
if the regression of hydrodynamic derivatives in the manoeuvring model should work well. However, filtering with the
EKF also needs an accurate manoeuvring model as the system model. Therefore the accurate manoeuvring model is both the input and
output of the parameter estimation. The system model in the EKF is guessed to solve this dilemma. A linear manoeuvring model
with hydrodynamic derivatives estimated with semi\sphinxhyphen{}empirical formulas is used as the initial guess. Once the
regressed manoeuvring model has been obtained, the parameter estimation can be rerun using the regressed manoeuvring model as the system model
in the EKF, to obtain an even better manoeuvring model. This procedure can be repeated several times for improved
accuracy. Using semi\sphinxhyphen{}empirical formulas for the initially guessed manoeuvring model adds prior knowledge about the ship dynamics to the regression. When used with the recursive EKF, this method is an innovation compared to
other parameter estimation methods.

\sphinxAtStartPar
The iterative process is composed of two basic steps:
\begin{enumerate}
\sphinxsetlistlabels{\arabic}{enumi}{enumii}{}{)}%
\item {} 
\sphinxAtStartPar
First, hydrodynamic derivatives of a predefined format of manoeuvring model are initially guessed. The derivatives
are estimated with semi\sphinxhyphen{}empirical formulas for a linear manoeuvring model to make an initial guess. The manoeuvring model is
used in the EKF and RTS smoother to filter all the model tests. The manoeuvring models are assumed to have
Markov property, meaning that future states depend only on the current state. Then the filtered data
with estimated hidden states from all the model tests can be joined into a time\sphinxhyphen{}independent dataset
passed to the motion regression. The hydrodynamic derivatives are regressed on quasi\sphinxhyphen{}static forces
from inverse dynamics giving the identified nonlinear manoeuvring model.

\item {} 
\sphinxAtStartPar
Re\sphinxhyphen{}run the iteration in the previous step with EKF that use the identified manoeuvring model from the previous
step to replace the guessed system model in the initial stage, such as with AVMM or MAVMM. There
should be more trust in this model than in the guessed model, so the covariance matrices should be
updated.

\end{enumerate}

\sphinxAtStartPar
In the following section, the methods of inverse dynamics, regression and EKF used in the proposed PIT
method, and their connections, are presented in detail.

\sphinxAtStartPar
An example with simulation results from the steps in the iterative EKF is shown in \hyperref[\detokenize{01.01_method:iterations}]{Fig.\@ \ref{\detokenize{01.01_method:iterations}}}

\begin{figure}[H]
\centering
\capstart

\noindent\sphinxincludegraphics{{0}.pdf}
\caption{Simulation with: initial model, first and second iteration of the parameter estimation}\label{\detokenize{01.01_method:iterations}}\end{figure}


\subsection{Inverse dynamics and regression}
\label{\detokenize{03.01_inverse_dynamics:inverse-dynamics-and-regression}}\label{\detokenize{03.01_inverse_dynamics::doc}}
\sphinxAtStartPar
Each manoeuvring model has some hydrodynamic functions \(X_D(u,v,r,\delta,thrust)\), \(Y_D(u,v,r,\delta,thrust)\), \(N_D(u,v,r,\delta,thrust)\) that are defined as polynomials. The hydrodynamic derivatives in these polynomials can be identified with force regression of measured forces and moments. The measured forces and moments are usually taken from Captive Model Tests (CMT), Planar Motion Mechanism (PMM) tests or Virtual Captive Tests (VCT). When the ship is free in all degrees of freedom, as in the present model tests, only
motions are recorded however. Hence, forces and moments causing ship motions need to be estimated by
solving the inverse dynamics problem.
The inverse dynamics is solved by restructuring the system equation (\autoref{equation:02.01_VMMs:eqqsystem}) to get the hydrodynamics functions on the left\sphinxhyphen{}hand side. If the mass and inertia of the ship including added masses: \(X_{\dot{u}}\), \(Y_{\dot{v}}\), \(Y_{\dot{r}}\), \(N_{\dot{v}}\) and \(N_{\dot{r}}\), are known, the forces in Prime system can be calculated using \autoref{equation:03.01_inverse_dynamics:eqxd}, \autoref{equation:03.01_inverse_dynamics:eqyd} and \autoref{equation:03.01_inverse_dynamics:eqnd}.
\begin{equation}\label{equation:03.01_inverse_dynamics:eqxd}
\begin{split}\displaystyle \operatorname{X_{D}'}{\left(u',v',r',\delta,thrust' \right)} = - X_{\dot{u}}' \dot{u}' + \dot{u}' m' - m' r'^{2} x_{G}' - m' r' v'\end{split}
\end{equation}\begin{equation}\label{equation:03.01_inverse_dynamics:eqyd}
\begin{split}\displaystyle \operatorname{Y_{D}'}{\left(u',v',r',\delta,thrust' \right)} = - Y_{\dot{r}}' \dot{r}' - Y_{\dot{v}}' \dot{v}' + \dot{r}' m' x_{G}' + \dot{v}' m' + m' r' u'\end{split}
\end{equation}\begin{equation}\label{equation:03.01_inverse_dynamics:eqnd}
\begin{split}\displaystyle \operatorname{N_{D}'}{\left(u',v',r',\delta,thrust' \right)} = I_{z}' \dot{r}' - N_{\dot{r}}' \dot{r}' - N_{\dot{v}}' \dot{v}' + \dot{v}' m' x_{G}' + m' r' u' x_{G}'\end{split}
\end{equation}
\sphinxAtStartPar
An example of forces calculated with inverse dynamics from motions in a turning circle test can be seen in \hyperref[\detokenize{03.01_inverse_dynamics:fig-inverse}]{Fig.\@ \ref{\detokenize{03.01_inverse_dynamics:fig-inverse}}}. The forces have been converted to SI units.

\begin{sphinxuseclass}{cell}
\begin{sphinxuseclass}{tag_remove_input}
\end{sphinxuseclass}
\end{sphinxuseclass}
\begin{figure}[H]
\centering
\capstart

\noindent\sphinxincludegraphics{{1}.pdf}
\caption{Example of forces and moments calculated with inverse dynamics on data from a turning circle test.}\label{\detokenize{03.01_inverse_dynamics:fig-inverse}}\end{figure}

\sphinxAtStartPar
Finding the hydrodynamic derivatives can be defined as a linear regression problem:
\begin{equation}\label{equation:03.01_inverse_dynamics:eqregression}
\begin{split}y = X\gamma + \epsilon\end{split}
\end{equation}
\sphinxAtStartPar
The model for the hydrodynamic forces must first be assumed to be the polynomials in the MAVMM.
The label vector \(y\) and feature matrix \(X\) in the regression problem in \autoref{equation:03.01_inverse_dynamics:eqregression} can now be inserted. As an example: the labels in the regression of the surge degree of freedom can be calculated using the inverse dynamics force, expressed with primed units:
\begin{equation}\label{equation:03.01_inverse_dynamics:diff_eq_X_y}
\begin{split}\displaystyle y = - X_{\dot{u}} \dot{u}' + \dot{u}' m' - m' r'^{2} x_{G'} - m' r' v'\end{split}
\end{equation}
\sphinxAtStartPar
The feature matrix \(X\) is expressed as:
\begin{equation}\label{equation:03.01_inverse_dynamics:diff_eq_X_X}
\begin{split}\displaystyle X = \left[\begin{matrix}thrust' & u' & \delta^{2} & r'^{2} & u'^{2} & r' v'\end{matrix}\right]\end{split}
\end{equation}
\sphinxAtStartPar
The regressed hydrodynamic derivatives are stored in the \(\gamma\) vector:
\begin{equation}\label{equation:03.01_inverse_dynamics:diff_eq_X_beta}
\begin{split}\displaystyle \gamma = \left[\begin{matrix}X_{T}\\X_{u}\\X_{\delta\delta}\\X_{rr}\\X_{uu}\\X_{vr}\end{matrix}\right]\end{split}
\end{equation}
\sphinxAtStartPar
The hydrodynamic derivatives in the manoeuvring model are considered Gaussian random variables when conducting the Ordinary Least Squares (OLS) regression. The hydrodynamic derivatives in the manoeuvring model are usually taken as the mean value of each regressed random variable, being the most likely estimate. The regression result can be described with a Multivariate Gaussian Distribution, defined by the regression’s mean values and covariance matrix. Monte Carlo simulations can be conducted with this distribution to study alternative realizations of the regression.

\sphinxAtStartPar
Strong multicollinearity is a known problem for the manoeuvring models \cite{luo_parameter_2016}, \cite{wang_quantifying_2018}.
The thrust coefficient \(X_T\) in the hydrodynamic function \(X_D\) in \autoref{equation:02.01_VMMs:eqxabkowitz} introduces multicollinearity to the regression. This coefficient can instead be calculated from the thrust deduction factor \(t_{df}\):
\begin{equation}\label{equation:03.01_inverse_dynamics:eqXthrust}
\begin{split}\displaystyle X_{T} = 1 - t_{df}\end{split}
\end{equation}
\sphinxAtStartPar
The \(X_T\) coefficient is excluded from the regression by moving it to the left\sphinxhyphen{}hand side of the regression equation \autoref{equation:03.01_inverse_dynamics:eqregression}:
\begin{equation}\label{equation:03.01_inverse_dynamics:eqexclude}
\begin{split}y-X_T \cdot thrust = X \gamma + \epsilon\end{split}
\end{equation}
\sphinxAtStartPar
Rudder coefficients (\(Y_R\)) from \(Y_D\) equation \autoref{equation:02.01_VMMs:eqyabkowitz} such as \(Y_{\delta}\), \(Y_{\delta T}\) etc. have been excluded in the same way by assuming a connection with their \(N_D\) equation counterpart through the rudder lever arm \(x_r\):
\begin{equation}\label{equation:03.01_inverse_dynamics:eqyr}
\begin{split}\displaystyle Y_{R} = \frac{N_{R}}{x_{r'}}\end{split}
\end{equation}

\subsection{Extended Kalman Filter (EKF)}
\label{\detokenize{04.01_EK:extended-kalman-filter-ekf}}\label{\detokenize{04.01_EK::doc}}
\sphinxAtStartPar
It is possible to do an exact parameter identification on perfect (simulated) data with no noise (see Section \(\ref{\detokenize{06.40_results_inverse_dynamics::doc}}\)). However, such data from physical experiments does not exist in reality. The measured data will always contain process noise and measurement noise. In order to mitigate this, the data is preprocessed using an Extended Kalman filter (EKF) and Rauch Tung Striebel (RTS) smoother which are both presented below.


\subsubsection{The EKF recursive algorithm}
\label{\detokenize{04.01_EK:the-ekf-recursive-algorithm}}
\sphinxAtStartPar
EKF is an extension of the Kalman Filter (KF) to work on nonlinear systems such as the manoeuvring models. The basic idea is that noise can be disregarded if it does not make sense from a physical point of view. If noisy measurement data were perfectly correct, this would mean that the ship has many vibrations that must have originated from tremendous forces, considering the large mass of the ship. The prior understanding of model tests suggests that these forces are not present during the test. Therefore, the noise should be considered as measurement noise and should be removed. Low\sphinxhyphen{}pass filtering is a common way to remove noise, where motions above some cut\sphinxhyphen{}off frequencies are regarded as unphysical measurement noise. The problem with low\sphinxhyphen{}pass filter is that it is hard to know what cut\sphinxhyphen{}off frequency to choose, either too low: removing part of the signal, or too high: keeping some unfiltered measurement noise in the data. The Kalman filter has a system model that continuously estimates the system’s state that is run in parallel with the measurement data. The filter estimates the current state as a combination of the measurement data and the system model estimate based on belief in the data and the model. If the data has low noise, the estimate turns toward that data. Conversely, if the model gives very good predictions, then that estimate turns towards the model.

\sphinxAtStartPar
The system’s inverse dynamics require the entire states, including positions, velocities, and accelerations, to be known. Only positions are known from the measurements, which means that velocities and accelerations are hidden states that the EKF should estimate.
The state transition \(f(\mathbf{x},\mathbf{c})\) is taken from the manoeuvring model (\autoref{equation:02.01_VMMs:eqf}) to use the manoeuvring model as the EKF predictor.
The state of the system is observed (measured) with a linear observation model (\autoref{equation:04.01_EK:eqobserve}) where \(\mathbf{y}\) is the measured data \(\mathbf{H}\) is the observation matrix and \(\eta\) is measurement noise.
\begin{equation}\label{equation:04.01_EK:eqobserve}
\begin{split}\mathbf{y} = \mathbf{H} x + \eta\end{split}
\end{equation}
\sphinxAtStartPar
The used EKF recursive algorithm used is summarized in the pseudocode below \cite{brown_introduction_1997}.
\label{04.01_EK:ek-algorithm}
\begin{sphinxadmonition}{note}{Algorithm 3.1 (Discrete\sphinxhyphen{}time extended Kalman filter)}



\sphinxAtStartPar
\sphinxstylestrong{Inputs} Initial values: \(x_0\), \(P_0\), \(C_d\), \(R_d\), \(Q_d\), \(E_d\)

\sphinxAtStartPar
\sphinxstylestrong{Output} Estimated states: \(\hat{x}\), estimated state covariances \(\hat{P}\)
\begin{enumerate}
\sphinxsetlistlabels{\arabic}{enumi}{enumii}{}{.}%
\item {} 
\sphinxAtStartPar
Initial values:
\begin{enumerate}
\sphinxsetlistlabels{\arabic}{enumii}{enumiii}{}{.}%
\item {} 
\sphinxAtStartPar
\(\hat{x}[0] = x_0\)

\item {} 
\sphinxAtStartPar
\(\hat{P}[0] = P_0\)

\end{enumerate}

\item {} 
\sphinxAtStartPar
For \(k\) in \(n\) measurements (time steps)
\begin{enumerate}
\sphinxsetlistlabels{\arabic}{enumii}{enumiii}{}{.}%
\item {} 
\sphinxAtStartPar
KF gain
\begin{enumerate}
\sphinxsetlistlabels{\arabic}{enumiii}{enumiv}{}{.}%
\item {} 
\sphinxAtStartPar
\(K[k]=\hat{P}[k] C_d^T \left(C_d \hat{P}[k] C_d^T + R_d\right)^{-1}\)

\item {} 
\sphinxAtStartPar
\(I_{KC} = I_n - K[k] C_d\)

\end{enumerate}

\item {} 
\sphinxAtStartPar
Update
\begin{enumerate}
\sphinxsetlistlabels{\arabic}{enumiii}{enumiv}{}{.}%
\item {} 
\sphinxAtStartPar
State corrector
\(\hat{x}[k] = \hat{x}[k] + K[k] (y - C_d \hat{x}[k]) \)

\item {} 
\sphinxAtStartPar
Covariance corrector
\(\hat{P}[k] = I_{KC} \cdot \hat{P}[k] I_{KC}^T + K[k] R_d K^T \)

\end{enumerate}

\item {} 
\sphinxAtStartPar
Predict
\begin{enumerate}
\sphinxsetlistlabels{\arabic}{enumiii}{enumiv}{}{.}%
\item {} 
\sphinxAtStartPar
State predictor
\(\hat{x}[k+1] = \hat{x}[k] + h \cdot \hat{f}(\hat{x}[k], c[k])\)

\item {} 
\sphinxAtStartPar
Covariance predictor
\(\hat{P}[k+1] = A_d[k]  \hat{P}[k] A_d[k]^T + E_d Q_d E_d^T \)

\end{enumerate}

\end{enumerate}

\end{enumerate}
\end{sphinxadmonition}

\sphinxAtStartPar
Where \(n\) is number of states (6 in this case), \(I_n\) is an \(n \cdot n\) identity matrix.
The transition matrix is calculated for each iteration using a Jacobian of the transition model:
\begin{equation}\label{equation:04.01_EK:eqjacobi}
\begin{split}A_d[k] = I + h \left. \frac{\partial f \left(x[k],c[k] \right)}{\partial x[k]} \right|_{x[k]=\hat{x}[k]}\end{split}
\end{equation}
\sphinxAtStartPar
This part and the fact that the nonlinear transition model is used directly as the predictor are the extension part of the EKF compared to the linear KF. Please note the linear approximation in \autoref{equation:04.01_EK:eqjacobi} around the current state. This approximation can cause stability problems if the real system and the linearized system deviates too much, when large time steps are used on a very nonlinear system. The Unscented Kalman Filter, which was used in \(\ref{revestido_herrero_two-step_2012}\), is an alternative that can be used in these situations.

\sphinxAtStartPar
The output from the filter contains the estimated states: \(\hat{x}\) and estimated state covariance matrix \(\hat{P}\). \(\hat{x}\) represent the most likely estimates, but the estimates have uncertainty that is expressed in \(\hat{P}\).
The state of the system is described by the ships position, heading, velocities and yaw velocity:
\begin{equation}\label{equation:04.01_EK:eqstates}
\begin{split}x = [x_0,y_0,\psi,u,v,r]^T\end{split}
\end{equation}
\sphinxAtStartPar
The initial state \(x_0\) is taken as the mean value of the first five measurements, where the velocities are estimated with numeric differentiation.

\sphinxAtStartPar
\(C_d\) selects the measured states (\(x_0\), \(y_0\), \(\psi\)):
\begin{equation}\label{equation:04.01_EK:eqcd}
\begin{split}\displaystyle C_{d} = h \left[\begin{matrix}1 & 0 & 0 & 0 & 0 & 0\\0 & 1 & 0 & 0 & 0 & 0\\0 & 0 & 1 & 0 & 0 & 0\end{matrix}\right]\end{split}
\end{equation}
\sphinxAtStartPar
\(E_d\) selects the hidden states (\(u\), \(v\), \(r\)):
\begin{equation}\label{equation:04.01_EK:eqed}
\begin{split}\displaystyle E_{d} = h \left[\begin{matrix}0 & 0 & 0\\0 & 0 & 0\\0 & 0 & 0\\1 & 0 & 0\\0 & 1 & 0\\0 & 0 & 1\end{matrix}\right]\end{split}
\end{equation}
\sphinxAtStartPar
Where \(h\) is the discrete time step, \(R_d\) describes the covariance matrix of the measurement, \(Q_d\) is the covariance matrix of the process model, and \(P_0\) is the initial state covariance.
Selecting good values for these three matrices is the most complicated part of getting the EKF to work well. The amount of expected measurement noise in the data should be inserted in to \(R_d\), and the amount of error generated by the process model (manoeuvring model) needs to be estimated in \(Q_d\). The choices for these matrices depend on the reliability of the present data and the present process model.


\subsubsection{The Rauch Tung Striebel (RTS) smoother}
\label{\detokenize{04.01_EK:the-rauch-tung-striebel-rts-smoother}}\label{\detokenize{04.01_EK:rts}}
\sphinxAtStartPar
The EKF is recursive and can be run online, continuously making new estimates as new measurements arrive. The EKF uses passed measurements to estimate states in the near future. This prediction is helpful for online applications such as  autopilots or autonomous ships (USVs) (USVs). This restriction is  unnecessary for the parameter estimation on already existing data where a whole time series of existing measurements are available. The fact that both past and future data are known can be used to improve the filter. An EKF filter can include future time steps by adding a smoother after the filter. The parameter estimation uses a Rauch Tung Striebel (RTS) smoother \cite{rauch_maximum_1965}, which is an algorithm that runs the EKF backward to also account for future time steps.
The EKF and RTS have been run on simulated data with Gaussian noise added to see if the real states can be identified. Results from this can be seen in  \hyperref[\detokenize{04.01_EK:fig-ekf}]{Fig.\@ \ref{\detokenize{04.01_EK:fig-ekf}}}. This figure shows that the RTS smoother is also needed to get an accurate estimate of the yaw acceleration.

\begin{figure}[H]
\centering
\capstart

\noindent\sphinxincludegraphics{{2}.pdf}
\caption{EKF and RTS on simulated data (real) with Gaussian noise added (raw).}\label{\detokenize{04.01_EK:fig-ekf}}\end{figure}


\subsection{Model development process}
\label{\detokenize{04.10_model_development_process:id1}}\label{\detokenize{04.10_model_development_process::doc}}
\sphinxAtStartPar
The general aim of developing a manoeuvring model with parameter estimation is to develop a model that can generalize outside the known data: the model should have good generalization. The method presented in this paper is evaluated with the ‘’Hold\sphinxhyphen{}Out’’ method where the data is divided into three sets:
\begin{itemize}
\item {} 
\sphinxAtStartPar
training set

\item {} 
\sphinxAtStartPar
validation set

\item {} 
\sphinxAtStartPar
test set.

\end{itemize}

\sphinxAtStartPar
The best manoeuvring model is selected from a selection of candidate models. The purpose with the training set is to train all the candidate models using the proposed parameter estimation method. The validation set is then used to select which one of the candidate models is the best. The training and validation sets are then joined to train the selected model as the final model, to be used in predicting the test set, which is used to evaluate the accuracy of the model. These three sets are not divided randomly, but rather to assess the model’s extrapolation ability. The data sets are therefore split to have the smallest: yaw rates, drift\sphinxhyphen{} and rudder\sphinxhyphen{}angles in the training set, the medium values in the validation set and the largest values in the test set.


\section{Presentation of Case Studies}
\label{\detokenize{05.01_case_studies:presentation-of-case-studies}}\label{\detokenize{05.01_case_studies:case-studies}}\label{\detokenize{05.01_case_studies::doc}}
\sphinxAtStartPar
The two case study model test results from the wPCC as shown in \hyperref[\detokenize{05.01_case_studies:wpcc-mdl}]{Fig.\@ \ref{\detokenize{05.01_case_studies:wpcc-mdl}}} and the well\sphinxhyphen{}known KVLCC2 are used to validate the proposed system identification method.
The models are developed following the process as described in section \(\ref{\detokenize{04.10_model_development_process:id1}}\). Consequently, both test cases aim to predict turning circle maneuvers. The main dimensions of the two case study ship models are listed in \hyperref[\detokenize{05.01_case_studies:main-dimensions-of-test-case-ship-models}]{Table \ref{\detokenize{05.01_case_studies:main-dimensions-of-test-case-ship-models}}}, with explanations in \hyperref[\detokenize{05.01_case_studies:list-of-main-dimensions-symbols}]{Table \ref{\detokenize{05.01_case_studies:list-of-main-dimensions-symbols}}}. The wPCC is a wind\sphinxhyphen{}powered car carrier tested at SSPA \cite{alexandersson_wpcc_2022}. This twin screw ship with large rudders has good course stability and symmetric hydrodynamic manoeuvring forces. The KVLCC2 model test data from the Hamburg ship model basin (HSVA) and Maritime Research Institute Netherlands (MARIN) was made available by SIMMAN2008 conference \cite{stern_experience_2011}. This single screw ship is more course unstable than the wPCC test case, and manoeuvring forces are unsymmetrical due to the single propeller. This instability makes it good as the second test case with parameter estimation on an unsymmetrical model.

\begin{figure}[H]
\centering
\capstart

\noindent\sphinxincludegraphics[height=150\sphinxpxdimen]{{wpcc_mdl}.png}
\caption{wPCC tested at SSPA. Copyright 2020 by SSPA Sweden AB.}\label{\detokenize{05.01_case_studies:wpcc-mdl}}\end{figure}


\begin{savenotes}\sphinxattablestart
\centering
\sphinxcapstartof{table}
\sphinxthecaptionisattop
\sphinxcaption{main dimensions of test case ship models}\label{\detokenize{05.01_case_studies:main-dimensions-of-test-case-ship-models}}
\sphinxaftertopcaption
\begin{tabulary}{\linewidth}[t]{|T|T|T|T|T|T|T|T|T|T|T|T|T|T|}
\hline
\sphinxstyletheadfamily &\sphinxstyletheadfamily 
\sphinxAtStartPar
\(B\) \([m]\)
&\sphinxstyletheadfamily 
\sphinxAtStartPar
\(D\) \([m]\)
&\sphinxstyletheadfamily 
\sphinxAtStartPar
\(L\) \([m]\)
&\sphinxstyletheadfamily 
\sphinxAtStartPar
\(L_{CG}\) \([m]\)
&\sphinxstyletheadfamily 
\sphinxAtStartPar
\(N_p\)
&\sphinxstyletheadfamily 
\sphinxAtStartPar
\(T\) \([m]\)
&\sphinxstyletheadfamily 
\sphinxAtStartPar
\(\alpha\)
&\sphinxstyletheadfamily 
\sphinxAtStartPar
\(\nabla\) \([m^3]\)
&\sphinxstyletheadfamily 
\sphinxAtStartPar
\(k_{zz}\)
&\sphinxstyletheadfamily 
\sphinxAtStartPar
\(m\) \([kg]\)
&\sphinxstyletheadfamily 
\sphinxAtStartPar
\(w_{p0}\)
&\sphinxstyletheadfamily 
\sphinxAtStartPar
\(x_{p}\) \([m]\)
&\sphinxstyletheadfamily 
\sphinxAtStartPar
\(x_{r}\) \([m]\)
\\
\hline
\sphinxAtStartPar
WPCC
&
\sphinxAtStartPar
0.95
&
\sphinxAtStartPar
0.12
&
\sphinxAtStartPar
5.01
&
\sphinxAtStartPar
0.0
&
\sphinxAtStartPar
2
&
\sphinxAtStartPar
0.21
&
\sphinxAtStartPar
41.2
&
\sphinxAtStartPar
0.44
&
\sphinxAtStartPar
0.25
&
\sphinxAtStartPar
441
&
\sphinxAtStartPar
0.15
&
\sphinxAtStartPar
\sphinxhyphen{}2.42
&
\sphinxAtStartPar
\sphinxhyphen{}2.42
\\
\hline
\sphinxAtStartPar
KVLCC2 (HSVA)
&
\sphinxAtStartPar
1.27
&
\sphinxAtStartPar
0.2
&
\sphinxAtStartPar
7.0
&
\sphinxAtStartPar
0.24
&
\sphinxAtStartPar
1
&
\sphinxAtStartPar
0.46
&
\sphinxAtStartPar
45.7
&
\sphinxAtStartPar
3.27
&
\sphinxAtStartPar
0.25
&
\sphinxAtStartPar
3272
&
\sphinxAtStartPar
0.4
&
\sphinxAtStartPar
\sphinxhyphen{}3.39
&
\sphinxAtStartPar
\sphinxhyphen{}3.5
\\
\hline
\end{tabulary}
\par
\sphinxattableend\end{savenotes}


\begin{savenotes}\sphinxattablestart
\centering
\sphinxcapstartof{table}
\sphinxthecaptionisattop
\sphinxcaption{List of main dimensions symbols}\label{\detokenize{05.01_case_studies:list-of-main-dimensions-symbols}}
\sphinxaftertopcaption
\begin{tabulary}{\linewidth}[t]{|T|T|}
\hline
\sphinxstyletheadfamily 
\sphinxAtStartPar
symbol
&\sphinxstyletheadfamily 
\sphinxAtStartPar
description
\\
\hline
\sphinxAtStartPar
\(B\)
&
\sphinxAtStartPar
Beam
\\
\hline
\sphinxAtStartPar
\(D\)
&
\sphinxAtStartPar
Propeller diameter
\\
\hline
\sphinxAtStartPar
\(L\)
&
\sphinxAtStartPar
Length between perpendiculars
\\
\hline
\sphinxAtStartPar
\(L_{CG}\)
&
\sphinxAtStartPar
Distance \(L/2\) to centre of gravity
\\
\hline
\sphinxAtStartPar
\(N_p\)
&
\sphinxAtStartPar
Number of propellers
\\
\hline
\sphinxAtStartPar
\(T\)
&
\sphinxAtStartPar
Draught
\\
\hline
\sphinxAtStartPar
\(\alpha\)
&
\sphinxAtStartPar
Scale factor
\\
\hline
\sphinxAtStartPar
\(\nabla\)
&
\sphinxAtStartPar
Volume displacement
\\
\hline
\sphinxAtStartPar
\(k_{zz}\)
&
\sphinxAtStartPar
Radius of gyration / \(L\)
\\
\hline
\sphinxAtStartPar
\(m\)
&
\sphinxAtStartPar
Mass (excluding added mass)
\\
\hline
\sphinxAtStartPar
\(w_{p0}\)
&
\sphinxAtStartPar
Wake fraction
\\
\hline
\sphinxAtStartPar
\(x_{p}\)
&
\sphinxAtStartPar
Longitudinal position of propeller
\\
\hline
\sphinxAtStartPar
\(x_{r}\)
&
\sphinxAtStartPar
Longitudinal position of rudder
\\
\hline
\end{tabulary}
\par
\sphinxattableend\end{savenotes}

\sphinxAtStartPar
The parameter estimation method requires an initial guessed linear manoeuvring model. For these initial models for the two test cases, their hydrodynamic derivatives are calculated with semi\sphinxhyphen{}empirical formulas (\autoref{equation:05.01_case_studies:eqnr}\sphinxhyphen{}\autoref{equation:05.01_case_studies:eqyvdot}) taken from \cite{brix_manoeuvring_1993} and shown in \hyperref[\detokenize{05.01_case_studies:intial-guess}]{Table \ref{\detokenize{05.01_case_studies:intial-guess}}}.
\begin{equation}\label{equation:05.01_case_studies:eqnr}
\begin{split}\displaystyle N_{r} = - \frac{\pi T^{2} \left(\frac{0.039 B}{T} - \frac{0.56 B}{L} + 0.25\right)}{L^{2}}\end{split}
\end{equation}\begin{equation}\label{equation:05.01_case_studies:eqnrdot}
\begin{split}\displaystyle N_{\dot{r}}' = - \frac{\pi T^{2} \left(\frac{0.017 B CB}{T} - \frac{0.33 B}{L} + 0.0833333333333333\right)}{L^{2}}\end{split}
\end{equation}\begin{equation}\label{equation:05.01_case_studies:eqnv}
\begin{split}\displaystyle N_{v} = - \frac{\pi T^{2} \left(0.5 + \frac{2.4 T}{L}\right)}{L^{2}}\end{split}
\end{equation}\begin{equation}\label{equation:05.01_case_studies:eqnvdot}
\begin{split}\displaystyle N_{\dot{v}}' = - \frac{\pi T^{2} \left(- \frac{0.04 B}{T} + \frac{1.1 B}{L}\right)}{L^{2}}\end{split}
\end{equation}\begin{equation}\label{equation:05.01_case_studies:eqxudot}
\begin{split}\displaystyle X_{\dot{u}}' = \frac{2.0 m}{L^{3} \rho \left(\pi \sqrt{\frac{L^{3}}{volume}} - 14\right)}\end{split}
\end{equation}\begin{equation}\label{equation:05.01_case_studies:eqyr}
\begin{split}\displaystyle Y_{r} = - \frac{\pi T^{2} \left(- \frac{0.08 B}{T} + \frac{2.2 B}{L} - 0.5\right)}{L^{2}}\end{split}
\end{equation}\begin{equation}\label{equation:05.01_case_studies:eqyrdot}
\begin{split}\displaystyle Y_{\dot{r}}' = - \frac{\pi T^{2} \left(- \frac{0.0033 B^{2}}{T^{2}} + \frac{0.67 B}{L}\right)}{L^{2}}\end{split}
\end{equation}\begin{equation}\label{equation:05.01_case_studies:eqyv}
\begin{split}\displaystyle Y_{v} = - \frac{\pi T^{2} \left(\frac{0.4 B CB}{T} + 1\right)}{L^{2}}\end{split}
\end{equation}\begin{equation}\label{equation:05.01_case_studies:eqyvdot}
\begin{split}\displaystyle Y_{\dot{v}}' = - \frac{\pi T^{2} \left(- \frac{5.1 B^{2}}{L^{2}} + \frac{0.16 B CB}{T} + 1\right)}{L^{2}}\end{split}
\end{equation}

\begin{savenotes}\sphinxattablestart
\centering
\sphinxcapstartof{table}
\sphinxthecaptionisattop
\sphinxcaption{Initial guessed derivatives in linear models (times 1000)}\label{\detokenize{05.01_case_studies:intial-guess}}
\sphinxaftertopcaption
\begin{tabulary}{\linewidth}[t]{|T|T|T|T|T|T|T|T|T|T|T|T|}
\hline
\sphinxstyletheadfamily &\sphinxstyletheadfamily 
\sphinxAtStartPar
\( N_{\delta} \)
&\sphinxstyletheadfamily 
\sphinxAtStartPar
\( N_{r} \)
&\sphinxstyletheadfamily 
\sphinxAtStartPar
\( N_{\dot{r}}' \)
&\sphinxstyletheadfamily 
\sphinxAtStartPar
\( N_{v} \)
&\sphinxstyletheadfamily 
\sphinxAtStartPar
\( N_{\dot{v}}' \)
&\sphinxstyletheadfamily 
\sphinxAtStartPar
\( X_{\dot{u}}' \)
&\sphinxstyletheadfamily 
\sphinxAtStartPar
\( Y_{\delta} \)
&\sphinxstyletheadfamily 
\sphinxAtStartPar
\( Y_{r} \)
&\sphinxstyletheadfamily 
\sphinxAtStartPar
\( Y_{\dot{r}}' \)
&\sphinxstyletheadfamily 
\sphinxAtStartPar
\( Y_{v} \)
&\sphinxstyletheadfamily 
\sphinxAtStartPar
\( Y_{\dot{v}}' \)
\\
\hline
\sphinxAtStartPar
WPCC
&
\sphinxAtStartPar
\sphinxhyphen{}1.5
&
\sphinxAtStartPar
\sphinxhyphen{}1.719
&
\sphinxAtStartPar
\sphinxhyphen{}0.299
&
\sphinxAtStartPar
\sphinxhyphen{}3.184
&
\sphinxAtStartPar
\sphinxhyphen{}0.128
&
\sphinxAtStartPar
0.179
&
\sphinxAtStartPar
3.0
&
\sphinxAtStartPar
2.402
&
\sphinxAtStartPar
\sphinxhyphen{}0.303
&
\sphinxAtStartPar
\sphinxhyphen{}9.713
&
\sphinxAtStartPar
\sphinxhyphen{}6.109
\\
\hline
\sphinxAtStartPar
KVLCC2 (HSVA)
&
\sphinxAtStartPar
\sphinxhyphen{}1.5
&
\sphinxAtStartPar
\sphinxhyphen{}3.415
&
\sphinxAtStartPar
\sphinxhyphen{}0.822
&
\sphinxAtStartPar
\sphinxhyphen{}8.707
&
\sphinxAtStartPar
\sphinxhyphen{}1.166
&
\sphinxAtStartPar
1.05
&
\sphinxAtStartPar
3.0
&
\sphinxAtStartPar
4.305
&
\sphinxAtStartPar
\sphinxhyphen{}1.271
&
\sphinxAtStartPar
\sphinxhyphen{}25.266
&
\sphinxAtStartPar
\sphinxhyphen{}15.846
\\
\hline
\end{tabulary}
\par
\sphinxattableend\end{savenotes}


\subsection{The wPCC test scenarios}
\label{\detokenize{05.01_case_studies:the-wpcc-test-scenarios}}
\sphinxAtStartPar
The wPCC test case focuses on predicting forces and moments from the ship hull and rudders. The propeller force is not part of the prediction model but is taken from the model test measurements.
The model test data used for modeling is split into training, validation and test data sets, following the model development process as described in section \(\ref{\detokenize{04.10_model_development_process:id1}}\). The training dataset contains self\sphinxhyphen{}propulsion, pull\sphinxhyphen{}out tests, and zigzag10/10 tests to starboard and port. The validation dataset consists of three zigzag20/20 tests and the turning circle test is left for the test set as shown in \hyperref[\detokenize{05.01_case_studies:fig-traintest}]{Fig.\@ \ref{\detokenize{05.01_case_studies:fig-traintest}}}.
If the manoeuvring model built by the proposed method based on a series of model tests including ZigZag10/10, 20/20 to port and starboard as well as self\sphinxhyphen{}propulsion and pull out test \cite{imo_standards_2002} can predict the turning circle maneuver, then it is a capable model.

\begin{figure}[H]
\centering
\capstart

\noindent\sphinxincludegraphics{{3}.pdf}
\caption{wPCC training, validation and testing datasets.}\label{\detokenize{05.01_case_studies:fig-traintest}}\end{figure}


\subsection{The KVLCC2 test scenarios}
\label{\detokenize{05.01_case_studies:the-kvlcc2-test-scenarios}}
\sphinxAtStartPar
The proposed method is also validated using the KVLCC2 case study ship model.
The propeller is part of the manoeuvring model for this test case, instead of only the hull and rudders, as in the wPCC test case, so that the entire ship can be simulated without additional input.
The model development process as described in section \(\ref{\detokenize{04.10_model_development_process:id1}}\) is applied for the KVLCC2 as well.
Here the training dataset contains various zigzag tests to starboard and port from model tests carried out at HSVA for the SIMMAN2008 conference \cite{stern_experience_2011}, where the ZigZag35/5 test is kept for the validation set. The test set is taken from turning circle model tests carried out at MARIN for the SIMMAN2008 conference \cite{stern_experience_2011} as seen in \hyperref[\detokenize{05.01_case_studies:fig-kvlcc2-traintest}]{Fig.\@ \ref{\detokenize{05.01_case_studies:fig-kvlcc2-traintest}}}.

\begin{figure}[H]
\centering
\capstart

\noindent\sphinxincludegraphics{{4}.pdf}
\caption{KVLCC2 training, validation and testing datasets.}\label{\detokenize{05.01_case_studies:fig-kvlcc2-traintest}}\end{figure}


\section{Results}
\label{\detokenize{06.01_results:results}}\label{\detokenize{06.01_results:id1}}\label{\detokenize{06.01_results::doc}}
\sphinxAtStartPar
The results motivating the choices of methods in the proposed parameter estimation are presented below. Result with the inverse dynamics regression is presented in Section \(\ref{\detokenize{06.40_results_inverse_dynamics::doc}}\) for one ideal case without measurement noise. A comparison between the proposed preprocessors, EKF and RTS, and alternative low\sphinxhyphen{}pass filter is presented in Section \(\ref{\detokenize{06.31_results_noise::doc}}\). Results with the parameter estimation for the turning circle test cases are presented for both ships in Section \(\ref{\detokenize{06.10_results_wpcc::doc}}\) and \(\ref{\detokenize{06.20_results_kvlcc2::doc}}\). Results from the KVLCC2 propeller model is also presented in Section  \(\ref{\detokenize{06.20_results_kvlcc2:the-kvlcc2-propeller-model}}\).


\subsection{Inverse dynamics}
\label{\detokenize{06.40_results_inverse_dynamics:inverse-dynamics}}\label{\detokenize{06.40_results_inverse_dynamics::doc}}
\begin{sphinxuseclass}{cell}
\begin{sphinxuseclass}{tag_remove_input}
\end{sphinxuseclass}
\end{sphinxuseclass}
\sphinxAtStartPar
The hydrodynamic derivatives within the manoeuvring model can be identified exactly at ideal conditions for the parameter estimation with no measurement noise and a perfect estimator. For example, artificial data from a turning circle test can be simulated by a pre\sphinxhyphen{}defined/true manoeuvring model. The hydrodynamic derivatives within the manoeuvring model can be identified with the same values. Results from such a simulation is shown in \hyperref[\detokenize{06.40_results_inverse_dynamics:fig-bar-parameters}]{Fig.\@ \ref{\detokenize{06.40_results_inverse_dynamics:fig-bar-parameters}}} where the regression has identified the true values precisely.

\begin{figure}[H]
\centering
\capstart

\noindent\sphinxincludegraphics{{5}.pdf}
\caption{True and regressed hydrodynamic derivatives in MAVMM identified with Inverse dynamics and OLS regression on a simulated turning circle with MAVMM.}\label{\detokenize{06.40_results_inverse_dynamics:fig-bar-parameters}}\end{figure}


\subsection{Preprocessing}
\label{\detokenize{06.31_results_noise:preprocessing}}\label{\detokenize{06.31_results_noise::doc}}
\sphinxAtStartPar
The low\sphinxhyphen{}pass filter is a prevalent alternative to preprocessing the model test data, as opposed to the EKF used by the proposed parameter estimation.
In order to study which of the filters works best, the proposed parameter estimation has been run on the wPCC model test data with the EKF + RTS smoother replaced by a Low\sphinxhyphen{}pass filter instead. The low\sphinxhyphen{}pass filter applies a first\sphinxhyphen{}order linear digital Butterworth filter twice, once forward and once backward, to get zero phase \cite{virtanen_scipy_2020}. \hyperref[\detokenize{06.31_results_noise:fig-lowpass-accuracy}]{Fig.\@ \ref{\detokenize{06.31_results_noise:fig-lowpass-accuracy}}} shows the average simulation error \( \overline{RMSE} \) with low\sphinxhyphen{}pass filters at various cut\sphinxhyphen{}off frequencies for all wPCC model tets. Corresponding error with parameter estimation using EKF + RTS is also shown in the figure. The simulation error for each model test is expressed as Root Mean Square Error \(RMSE\) (\autoref{equation:06.31_results_noise:eqrmse}) of the distance between the position from the model test and simulation.
\begin{equation}\label{equation:06.31_results_noise:eqrmse}
\begin{split}RMSE=\sqrt{ \frac{\sum_{n=1}^{N} (d_n^2) }{N}} \end{split}
\end{equation}
\sphinxAtStartPar
where \(d_n\) is the euclidean distance for each time step between the model test positions (\(x_0\), \(y_0\)) and the predicted positions.

\begin{figure}[H]
\centering
\capstart

\noindent\sphinxincludegraphics{{6}.pdf}
\caption{Average simulation error with MAVMM fitted on wPCC model test data using low\sphinxhyphen{}pass filters with various cutt off frequency or EKF.}\label{\detokenize{06.31_results_noise:fig-lowpass-accuracy}}\end{figure}

\sphinxAtStartPar
Even though high accuracy can be obtained using a low\sphinxhyphen{}pass filter as the pre\sphinxhyphen{}processor, if an optimal cut\sphinxhyphen{}off frequency is selected, its accuracy decreases quickly at lower or higher frequencies. With higher cut\sphinxhyphen{}off frequencies, too much of the measurement error remains in the data, resulting in poor performance of the OLS regression. In extreme cases, it is like having no filter at all. Using too low of a cut\sphinxhyphen{}off frequency removes too much, including parts of the actual signal. The results show that the low\sphinxhyphen{}pass filter with a 7 Hz cut\sphinxhyphen{}off frequency has the lowest error among the low\sphinxhyphen{}pass filters, but EKF + RTS in the parameter estimation has an even lower error, which is why this is used as the preprocessor in the proposed parameter estimation.


\subsection{The wPCC tests}
\label{\detokenize{06.10_results_wpcc:the-wpcc-tests}}\label{\detokenize{06.10_results_wpcc::doc}}
\sphinxAtStartPar
The LVMM was ruled too simple, so only the AVMM and MAVMM were considered possible manoeuvring models in the cross\sphinxhyphen{}validation.
Forces and moment predicted for the validation dataset with the manoeuvring models fitted with proposed parameter estimation on the training set are shown in \hyperref[\detokenize{06.10_results_wpcc:fig-validation-forces}]{Fig.\@ \ref{\detokenize{06.10_results_wpcc:fig-validation-forces}}}. It can be seen that the fitted AVMM overpredicts the forces by far. Therefore, simulations of the validation cases are only possible using the MAVMM, which is selected as the suitable manoeuvring model for the wPCC.
The simulations are shown for one of the ZigZag20/20 validation cases in \hyperref[\detokenize{06.10_results_wpcc:fig-validation-sim}]{Fig.\@ \ref{\detokenize{06.10_results_wpcc:fig-validation-sim}}}.

\begin{figure}[H]
\centering
\capstart

\noindent\sphinxincludegraphics{{7}.pdf}
\caption{Validation of force models for wPCC ZigZag20/20.}\label{\detokenize{06.10_results_wpcc:fig-validation-forces}}\end{figure}

\begin{figure}[H]
\centering
\capstart

\noindent\sphinxincludegraphics{{8}.pdf}
\caption{Validation with simulations for wPCC ZigZag20/20.}\label{\detokenize{06.10_results_wpcc:fig-validation-sim}}\end{figure}

\sphinxAtStartPar
The over\sphinxhyphen{}prediction of forces with the AVMM can be explained by the large problems with multicollinearity that were encountered when applying the parameter estimation method to the wPCC data. The absolute correlation coefficient between the features in the wPCC yaw moment regression are shown in \hyperref[\detokenize{06.10_results_wpcc:fig-ncorr}]{Fig.\@ \ref{\detokenize{06.10_results_wpcc:fig-ncorr}}}. It can be seen that most of the coefficients have very high absolute correlation (indicated in black). Some of the regressed hydrodynamic derivatives in the AVMM also have a substantial values and large uncertainty.

\begin{figure}[H]
\centering
\capstart

\noindent\sphinxincludegraphics{{9}.pdf}
\caption{Absolute correlation between the features in the wPCC yaw moment regression of AVMM}\label{\detokenize{06.10_results_wpcc:fig-ncorr}}\end{figure}

\sphinxAtStartPar
For the wPCC the prediction was conducted using simulation of the turning circle by the trained MAVMM, and the prediction results are presented in \hyperref[\detokenize{06.10_results_wpcc:fig-track-plot-testing-sim}]{Fig.\@ \ref{\detokenize{06.10_results_wpcc:fig-track-plot-testing-sim}}}, \hyperref[\detokenize{06.10_results_wpcc:fig-testing-sim}]{Fig.\@ \ref{\detokenize{06.10_results_wpcc:fig-testing-sim}}}. Monte Carlo simulations with alternative realizations of the regression, considering the uncertainty in the regressed parameters, are also shown in these figures. The alternative realizations have similar simulation results to the model with mean values of the regression (black line).
Advance and tactical diameter \cite{imo_standards_2002} differs 4\% and 1\% between prediction simulation and corresponding results from the model tests (\hyperref[\detokenize{06.10_results_wpcc:tab-wpcc-advance}]{Table \ref{\detokenize{06.10_results_wpcc:tab-wpcc-advance}}})
which are acceptable deviations for the wPCC considering the large margin to the limits of the IMO standard \cite{imo_standards_2002}.

\begin{figure}[H]
\centering
\capstart

\noindent\sphinxincludegraphics{{10}.pdf}
\caption{Turning circle test case for wPCC, track plots from model test and simulation.}\label{\detokenize{06.10_results_wpcc:fig-track-plot-testing-sim}}\end{figure}

\begin{figure}[H]
\centering
\capstart

\noindent\sphinxincludegraphics{{11}.pdf}
\caption{Turning circle test case for wPCC, time series from model test and simulation.}\label{\detokenize{06.10_results_wpcc:fig-testing-sim}}\end{figure}


\begin{savenotes}\sphinxattablestart
\centering
\sphinxcapstartof{table}
\sphinxthecaptionisattop
\sphinxcaption{wPCC Predicted turning circle advance and tactical diameter compared to SSPA model tests and IMO limit}\label{\detokenize{06.10_results_wpcc:tab-wpcc-advance}}
\sphinxaftertopcaption
\begin{tabulary}{\linewidth}[t]{|T|T|T|T|T|}
\hline
\sphinxstyletheadfamily &\sphinxstyletheadfamily 
\sphinxAtStartPar
Advance {[}m{]}
&\sphinxstyletheadfamily 
\sphinxAtStartPar
Advance (IMO) {[}m{]}
&\sphinxstyletheadfamily 
\sphinxAtStartPar
Tactical diameter {[}m{]}
&\sphinxstyletheadfamily 
\sphinxAtStartPar
Tactical diameter (IMO) {[}m{]}
\\
\hline
\sphinxAtStartPar
Model test
&
\sphinxAtStartPar
12.82
&
\sphinxAtStartPar
22.57
&
\sphinxAtStartPar
14.76
&
\sphinxAtStartPar
25.07
\\
\hline
\sphinxAtStartPar
Prediction
&
\sphinxAtStartPar
13.3
&
\sphinxAtStartPar
22.57
&
\sphinxAtStartPar
14.93
&
\sphinxAtStartPar
25.07
\\
\hline
\end{tabulary}
\par
\sphinxattableend\end{savenotes}

\sphinxAtStartPar
The mean values and standard error (se) of the hydrodynamic derivatives expressed with prime units for the wPCC obtained with parameter estimation of MAVMM (\autoref{equation:02.01_VMMs:eqxmartinssimple},\autoref{equation:02.01_VMMs:eqymartinssimple}, \autoref{equation:02.01_VMMs:eqnmartinssimple}) applied on all the wPCC data (including the turning circle)  are shown in \hyperref[\detokenize{06.10_results_wpcc:wpcc-derivatives}]{Table \ref{\detokenize{06.10_results_wpcc:wpcc-derivatives}}}.


\begin{savenotes}\sphinxattablestart
\centering
\sphinxcapstartof{table}
\sphinxthecaptionisattop
\sphinxcaption{wPCC MAVMM derivatives (prime units times 1000)}\label{\detokenize{06.10_results_wpcc:wpcc-derivatives}}
\sphinxaftertopcaption
\begin{tabulary}{\linewidth}[t]{|T|T|T|T|T|T|T|T|T|}
\hline
\sphinxstyletheadfamily 
\sphinxAtStartPar
name
&\sphinxstyletheadfamily 
\sphinxAtStartPar
mean
&\sphinxstyletheadfamily 
\sphinxAtStartPar
se
&\sphinxstyletheadfamily 
\sphinxAtStartPar
name
&\sphinxstyletheadfamily 
\sphinxAtStartPar
mean
&\sphinxstyletheadfamily 
\sphinxAtStartPar
se
&\sphinxstyletheadfamily 
\sphinxAtStartPar
name
&\sphinxstyletheadfamily 
\sphinxAtStartPar
mean
&\sphinxstyletheadfamily 
\sphinxAtStartPar
se
\\
\hline
\sphinxAtStartPar
\( X_{\delta\delta} \)
&
\sphinxAtStartPar
\sphinxhyphen{}2.927
&
\sphinxAtStartPar
0.011
&
\sphinxAtStartPar
\( Y_{ur} \)
&
\sphinxAtStartPar
\sphinxhyphen{}65.507
&
\sphinxAtStartPar
0.082
&
\sphinxAtStartPar
\( N_{\delta} \)
&
\sphinxAtStartPar
\sphinxhyphen{}1.993
&
\sphinxAtStartPar
0.002
\\
\hline
\sphinxAtStartPar
\( X_{vr} \)
&
\sphinxAtStartPar
\sphinxhyphen{}7.737
&
\sphinxAtStartPar
0.066
&
\sphinxAtStartPar
\( Y_{v} \)
&
\sphinxAtStartPar
\sphinxhyphen{}20.347
&
\sphinxAtStartPar
0.016
&
\sphinxAtStartPar
\( N_{T\delta} \)
&
\sphinxAtStartPar
\sphinxhyphen{}5.392
&
\sphinxAtStartPar
0.599
\\
\hline
\sphinxAtStartPar
\( X_{rr} \)
&
\sphinxAtStartPar
\sphinxhyphen{}1.413
&
\sphinxAtStartPar
0.026
&
\sphinxAtStartPar
\( Y_{u} \)
&
\sphinxAtStartPar
\sphinxhyphen{}0.027
&
\sphinxAtStartPar
0.001
&
\sphinxAtStartPar
\( N_{r} \)
&
\sphinxAtStartPar
\sphinxhyphen{}37.341
&
\sphinxAtStartPar
0.096
\\
\hline
\sphinxAtStartPar
\( X_{uu} \)
&
\sphinxAtStartPar
20.124
&
\sphinxAtStartPar
0.137
&
\sphinxAtStartPar
\( Y_{r} \)
&
\sphinxAtStartPar
64.14
&
\sphinxAtStartPar
0.083
&
\sphinxAtStartPar
\( N_{u} \)
&
\sphinxAtStartPar
\sphinxhyphen{}0.003
&
\sphinxAtStartPar
0.0
\\
\hline
\sphinxAtStartPar
\( X_{u} \)
&
\sphinxAtStartPar
\sphinxhyphen{}20.948
&
\sphinxAtStartPar
0.137
&&&&
\sphinxAtStartPar
\( N_{ur} \)
&
\sphinxAtStartPar
35.525
&
\sphinxAtStartPar
0.096
\\
\hline&&&&&&
\sphinxAtStartPar
\( N_{v} \)
&
\sphinxAtStartPar
\sphinxhyphen{}0.05
&
\sphinxAtStartPar
0.004
\\
\hline&&&&&&
\sphinxAtStartPar
\( N_{vv\delta} \)
&
\sphinxAtStartPar
\sphinxhyphen{}19.051
&
\sphinxAtStartPar
0.054
\\
\hline
\end{tabulary}
\par
\sphinxattableend\end{savenotes}


\subsection{The KVLCC2 tests}
\label{\detokenize{06.20_results_kvlcc2:the-kvlcc2-tests}}\label{\detokenize{06.20_results_kvlcc2::doc}}
\sphinxAtStartPar
The propeller is part of the manoeuvring model for thr KVLCC2 test case. A propeller prediction model needs to be regressed, based on thrust measurements from the model tests.


\subsubsection{The KVLCC2 propeller model}
\label{\detokenize{06.20_results_kvlcc2:the-kvlcc2-propeller-model}}\label{\detokenize{06.20_results_kvlcc2:results-propeller-model}}
\sphinxAtStartPar
The coefficients of \(K_T\) (\autoref{equation:02.10_propeller_model:eqkt}) were regressed from the KVLCC2 propeller characteristics from SIMMAN2008 HSVA model tests \cite{stern_experience_2011} (\(k_0\):\DUrole{output,text_plain}{0.32419}, \(k_1\):\DUrole{output,text_plain}{\sphinxhyphen{}0.22091}, \(k_2\):\DUrole{output,text_plain}{\sphinxhyphen{}0.14905}).
The Polynomial propeller model was developed with polynomial regression and cross\sphinxhyphen{}validation on the training and validation datasets to make the best feature selection.
A cross\sphinxhyphen{}validation study was carried out on the three candidate propeller models: the MMG propeller model, the simple propeller model, and the Polynomial propeller model. The training and validation sets were made of the entire model test time series from the HSVA model tests.
The model tests were divided into the test and validation sets randomly. The random training and validation were repeated 100 times. The Polynomial model was selected, having the highest accuracy. Taylor wake \(w_{p0}\) = \DUrole{output,text_plain}{0.4} was used in all three models, the MMG model used \(C_1\)=\DUrole{output,text_plain}{2.0}, \(C_2\)=\DUrole{output,text_plain}{1.6} when \(\beta_p>0\) and \(C_2\)=\DUrole{output,text_plain}{1.1} when \(\beta_p<=0\) \cite{yasukawa_introduction_2015-1}. \hyperref[\detokenize{06.20_results_kvlcc2:fig-propeller-validation}]{Fig.\@ \ref{\detokenize{06.20_results_kvlcc2:fig-propeller-validation}}} shows a small part of the cross\sphinxhyphen{}validation.

\begin{figure}[H]
\centering
\capstart

\noindent\sphinxincludegraphics{{12}.pdf}
\caption{Validation of MMG, Simple and Poynomial propeller models for KVLCC2.}\label{\detokenize{06.20_results_kvlcc2:fig-propeller-validation}}\end{figure}

\sphinxAtStartPar
\hyperref[\detokenize{06.20_results_kvlcc2:kvlcc2-propeller-model}]{Table \ref{\detokenize{06.20_results_kvlcc2:kvlcc2-propeller-model}}} shows coefficients of the polynomial propeller model fitted on the training and validation dataset for KVLCC2.


\begin{savenotes}\sphinxattablestart
\centering
\sphinxcapstartof{table}
\sphinxthecaptionisattop
\sphinxcaption{KVLCC2 propeller model}\label{\detokenize{06.20_results_kvlcc2:kvlcc2-propeller-model}}
\sphinxaftertopcaption
\begin{tabulary}{\linewidth}[t]{|T|T|T|}
\hline
\sphinxstyletheadfamily &\sphinxstyletheadfamily 
\sphinxAtStartPar
\(\beta_p>0\)
&\sphinxstyletheadfamily 
\sphinxAtStartPar
\(\beta_p<=0\)
\\
\hline
\sphinxAtStartPar
\(C_1\)
&
\sphinxAtStartPar
\sphinxhyphen{}0.1735
&
\sphinxAtStartPar
\sphinxhyphen{}0.1066
\\
\hline
\sphinxAtStartPar
\(C_2\)
&
\sphinxAtStartPar
0.4589
&
\sphinxAtStartPar
0.0771
\\
\hline
\sphinxAtStartPar
\(C_3\)
&
\sphinxAtStartPar
\sphinxhyphen{}1.8865
&
\sphinxAtStartPar
1.2958
\\
\hline
\sphinxAtStartPar
\(C_4\)
&
\sphinxAtStartPar
0.0515
&
\sphinxAtStartPar
0.0514
\\
\hline
\end{tabulary}
\par
\sphinxattableend\end{savenotes}


\subsubsection{KVLCC2 manoeuvring model}
\label{\detokenize{06.20_results_kvlcc2:kvlcc2-manoeuvring-model}}
\sphinxAtStartPar
The LVMM was ruled too simple, for KVLCC2 as well, so only the AVMM and MAVMM were considered possible manoeuvring models in the cross\sphinxhyphen{}validation.
The forces and moments applied on the hull, rudder, and propeller predicted with the AVMM and MAVMM fitted with the proposed parameter estimation on the training set are shown in \hyperref[\detokenize{06.20_results_kvlcc2:fig-kvlcc2-validation-forces}]{Fig.\@ \ref{\detokenize{06.20_results_kvlcc2:fig-kvlcc2-validation-forces}}}.
The forces are well predicted with both manoeuvring models. The AVMM is not giving the large over predictions that were seen for wPCC (see Section \(\ref{\detokenize{06.10_results_wpcc::doc}}\)). However, the MAVMM is still slightly better and is therefore selected as the suitable manoeuvring model for the KVLCC2.

\begin{figure}[H]
\centering
\capstart

\noindent\sphinxincludegraphics{{13}.pdf}
\caption{Validation of force models for KVLCC2.}\label{\detokenize{06.20_results_kvlcc2:fig-kvlcc2-validation-forces}}\end{figure}

\sphinxAtStartPar
Simulations of the validation cases with the MAVMM is shown for one of the ZigZag20/20 validation cases in \hyperref[\detokenize{06.20_results_kvlcc2:fig-kvlcc2-validation-sim}]{Fig.\@ \ref{\detokenize{06.20_results_kvlcc2:fig-kvlcc2-validation-sim}}} and \hyperref[\detokenize{06.20_results_kvlcc2:fig-kvlcc2-validation-sim-error}]{Fig.\@ \ref{\detokenize{06.20_results_kvlcc2:fig-kvlcc2-validation-sim-error}}} where the predicted thrust is also shown.

\begin{figure}[H]
\centering
\capstart

\noindent\sphinxincludegraphics{{14}.pdf}
\caption{Validation with simulations for KVLCC2.}\label{\detokenize{06.20_results_kvlcc2:fig-kvlcc2-validation-sim}}\end{figure}

\begin{figure}[H]
\centering
\capstart

\noindent\sphinxincludegraphics{{15}.pdf}
\caption{Validation error (prediction\sphinxhyphen{}model test) with simulations for KVLCC2.}\label{\detokenize{06.20_results_kvlcc2:fig-kvlcc2-validation-sim-error}}\end{figure}

\sphinxAtStartPar
Results from the final prediction of the turning circle test are shown in  \hyperref[\detokenize{06.20_results_kvlcc2:fig-kvlcc2-track-plot-testing-sim}]{Fig.\@ \ref{\detokenize{06.20_results_kvlcc2:fig-kvlcc2-track-plot-testing-sim}}}, \hyperref[\detokenize{06.20_results_kvlcc2:fig-kvlcc2-testing-sim}]{Fig.\@ \ref{\detokenize{06.20_results_kvlcc2:fig-kvlcc2-testing-sim}}} and \hyperref[\detokenize{06.20_results_kvlcc2:fig-kvlcc2-testing-sim-error}]{Fig.\@ \ref{\detokenize{06.20_results_kvlcc2:fig-kvlcc2-testing-sim-error}}}. The prediction is conducted using simulation with the MAVMM trained on the training and validation dataset. Monte Carlo simulations with alternative realizations of the regression are also shown in this figure. The alternative realizations are very similar to the model with mean values of the regression (black line).

\begin{figure}[H]
\centering
\capstart

\noindent\sphinxincludegraphics{{16}.pdf}
\caption{Comparison between predicted Turning circle test with MAVMM trained on HSVA data and MARIN model test results for KVLCC2.}\label{\detokenize{06.20_results_kvlcc2:fig-kvlcc2-track-plot-testing-sim}}\end{figure}

\begin{figure}[H]
\centering
\capstart

\noindent\sphinxincludegraphics{{17}.pdf}
\caption{Comparison between predicted Turning circle test with MAVMM trained on HSVA data and MARIN model test results for KVLCC2.}\label{\detokenize{06.20_results_kvlcc2:fig-kvlcc2-testing-sim}}\end{figure}

\begin{figure}[H]
\centering
\capstart

\noindent\sphinxincludegraphics{{18}.pdf}
\caption{The prediction error (prediction\sphinxhyphen{}model test) for Turning circle test with MAVMM trained on HSVA data and MARIN model test results for KVLCC2.}\label{\detokenize{06.20_results_kvlcc2:fig-kvlcc2-testing-sim-error}}\end{figure}

\sphinxAtStartPar
For KVLCC2 comparisons of turning circle advance and tactical diameter compared to the model test result is shown in \hyperref[\detokenize{06.20_results_kvlcc2:tab-kvlcc2-advance}]{Table \ref{\detokenize{06.20_results_kvlcc2:tab-kvlcc2-advance}}}. Predicted advance and tactical diameter differ 2\% and 5\%, which can be considered acceptable, considering the margin to the IMO standard limits, which are also shown in this table. The results are also closer to the model tests than a similar study conducted for the KVLCC2 \cite{he_nonparametric_2022}.


\begin{savenotes}\sphinxattablestart
\centering
\sphinxcapstartof{table}
\sphinxthecaptionisattop
\sphinxcaption{KVLCC2 Predicted turning circle advance (A) and tactical diameter (TD) compared to MARIN model tests and IMO limit}\label{\detokenize{06.20_results_kvlcc2:tab-kvlcc2-advance}}
\sphinxaftertopcaption
\begin{tabulary}{\linewidth}[t]{|T|T|T|T|T|T|T|}
\hline
\sphinxstyletheadfamily 
\sphinxAtStartPar
delta
&\sphinxstyletheadfamily 
\sphinxAtStartPar
A (model test) {[}m{]}
&\sphinxstyletheadfamily 
\sphinxAtStartPar
A (prediction) {[}m{]}
&\sphinxstyletheadfamily 
\sphinxAtStartPar
A (IMO) {[}m{]}
&\sphinxstyletheadfamily 
\sphinxAtStartPar
TD (model test) {[}m{]}
&\sphinxstyletheadfamily 
\sphinxAtStartPar
TD (prediction) {[}m{]}
&\sphinxstyletheadfamily 
\sphinxAtStartPar
TD (IMO) {[}m{]}
\\
\hline
\sphinxAtStartPar
35.0
&
\sphinxAtStartPar
21.59
&
\sphinxAtStartPar
21.21
&
\sphinxAtStartPar
31.5
&
\sphinxAtStartPar
21.72
&
\sphinxAtStartPar
23.07
&
\sphinxAtStartPar
35.0
\\
\hline
\sphinxAtStartPar
\sphinxhyphen{}35.0
&
\sphinxAtStartPar
22.54
&
\sphinxAtStartPar
22.1
&
\sphinxAtStartPar
31.5
&
\sphinxAtStartPar
23.55
&
\sphinxAtStartPar
24.29
&
\sphinxAtStartPar
35.0
\\
\hline
\end{tabulary}
\par
\sphinxattableend\end{savenotes}

\sphinxAtStartPar
The mean values and standard error (se) of the hydrodynamic derivatives expressed with prime units for the KVLCC2 obtained with parameter estimation of MAVMM (\(\autoref{equation:02.01_VMMs:eqxmartinssimple}\),\(\autoref{equation:02.01_VMMs:eqymartinssimple}\), \(\autoref{equation:02.01_VMMs:eqnmartinssimple}\)) applied on all the HSVA data are shown in \hyperref[\detokenize{06.20_results_kvlcc2:kvlcc2-derivatives}]{Table \ref{\detokenize{06.20_results_kvlcc2:kvlcc2-derivatives}}}.


\begin{savenotes}\sphinxattablestart
\centering
\sphinxcapstartof{table}
\sphinxthecaptionisattop
\sphinxcaption{KVLCC2 MAVMM derivatives (prime units times 1000)}\label{\detokenize{06.20_results_kvlcc2:kvlcc2-derivatives}}
\sphinxaftertopcaption
\begin{tabulary}{\linewidth}[t]{|T|T|T|T|T|T|T|T|T|}
\hline
\sphinxstyletheadfamily 
\sphinxAtStartPar
name
&\sphinxstyletheadfamily 
\sphinxAtStartPar
mean
&\sphinxstyletheadfamily 
\sphinxAtStartPar
se
&\sphinxstyletheadfamily 
\sphinxAtStartPar
name
&\sphinxstyletheadfamily 
\sphinxAtStartPar
mean
&\sphinxstyletheadfamily 
\sphinxAtStartPar
se
&\sphinxstyletheadfamily 
\sphinxAtStartPar
name
&\sphinxstyletheadfamily 
\sphinxAtStartPar
mean
&\sphinxstyletheadfamily 
\sphinxAtStartPar
se
\\
\hline
\sphinxAtStartPar
\( X_{vr} \)
&
\sphinxAtStartPar
\sphinxhyphen{}11.454
&
\sphinxAtStartPar
0.272
&
\sphinxAtStartPar
\( Y_{T} \)
&
\sphinxAtStartPar
77.34
&
\sphinxAtStartPar
1.23
&
\sphinxAtStartPar
\( N_{\delta} \)
&
\sphinxAtStartPar
\sphinxhyphen{}1.274
&
\sphinxAtStartPar
0.003
\\
\hline
\sphinxAtStartPar
\( X_{rr} \)
&
\sphinxAtStartPar
\sphinxhyphen{}1.406
&
\sphinxAtStartPar
0.068
&
\sphinxAtStartPar
\( Y_{r} \)
&
\sphinxAtStartPar
256.065
&
\sphinxAtStartPar
0.654
&
\sphinxAtStartPar
\( N_{r} \)
&
\sphinxAtStartPar
\sphinxhyphen{}105.618
&
\sphinxAtStartPar
0.179
\\
\hline
\sphinxAtStartPar
\( X_{\delta\delta} \)
&
\sphinxAtStartPar
\sphinxhyphen{}2.719
&
\sphinxAtStartPar
0.013
&
\sphinxAtStartPar
\( Y_{v} \)
&
\sphinxAtStartPar
\sphinxhyphen{}24.467
&
\sphinxAtStartPar
0.02
&
\sphinxAtStartPar
\( N_{T} \)
&
\sphinxAtStartPar
\sphinxhyphen{}32.523
&
\sphinxAtStartPar
0.274
\\
\hline
\sphinxAtStartPar
\( X_{uu} \)
&
\sphinxAtStartPar
80.508
&
\sphinxAtStartPar
0.618
&
\sphinxAtStartPar
\( Y_{ur} \)
&
\sphinxAtStartPar
\sphinxhyphen{}252.991
&
\sphinxAtStartPar
0.658
&
\sphinxAtStartPar
\( N_{u} \)
&
\sphinxAtStartPar
0.063
&
\sphinxAtStartPar
0.001
\\
\hline
\sphinxAtStartPar
\( X_{u} \)
&
\sphinxAtStartPar
\sphinxhyphen{}81.415
&
\sphinxAtStartPar
0.618
&
\sphinxAtStartPar
\( Y_{u} \)
&
\sphinxAtStartPar
\sphinxhyphen{}0.119
&
\sphinxAtStartPar
0.003
&
\sphinxAtStartPar
\( N_{v} \)
&
\sphinxAtStartPar
\sphinxhyphen{}7.156
&
\sphinxAtStartPar
0.016
\\
\hline&&&&&&
\sphinxAtStartPar
\( N_{T\delta} \)
&
\sphinxAtStartPar
\sphinxhyphen{}391.596
&
\sphinxAtStartPar
0.941
\\
\hline&&&&&&
\sphinxAtStartPar
\( N_{vv\delta} \)
&
\sphinxAtStartPar
\sphinxhyphen{}19.257
&
\sphinxAtStartPar
0.089
\\
\hline&&&&&&
\sphinxAtStartPar
\( N_{ur} \)
&
\sphinxAtStartPar
102.252
&
\sphinxAtStartPar
0.183
\\
\hline
\end{tabulary}
\par
\sphinxattableend\end{savenotes}


\subsection{Discussion}
\label{\detokenize{06.50_results_discussion:discussion}}\label{\detokenize{06.50_results_discussion::doc}}
\sphinxAtStartPar
Using inverse dynamics in the proposed parameter estimation can find the parameters in a manoeuvring model precisely when there is no measurement noise, and the selected manoeuvring model is a perfect model. This type of result can be seen when identifying parameters in a manoeuvring model on data from simulations with the same manoeuvring model. In order to succeed in system identification on actual model test data, measurement noise needs to be handled, and a manoeuvring model as close as possible to the real system should be used. The proposed parameter estimation method requires that the model test data is preprocessed to remove measurement noises. The proposed iterative EKF and RTS smoothers gives higher accuracy and avoids finding the optimal cut\sphinxhyphen{}off frequency for using a low\sphinxhyphen{}pass filter. The linearization in the EKF did not cause any stability problems, with the high frequency model test data used (100 Hz), which can be a problem for more sparse time series, with longer time steps. In this case, Unscented Kalman Filter (UKF) can be considered as an alternative.

\sphinxAtStartPar
Multicollinearity was a significant problem with the AVMM for both the wPCC and KVLCC2 data. Consequently, some of the regressed hydrodynamic derivatives in the AVMM have unphysically large values and substantial uncertainties. The model is still mathematically correct, where the regressed polynomials fit the training data well.
The regressed polynomial is the sum of large counteracting coefficients. The model works as long as the states are similar to the training data. However, when extrapolating, it is easy to imagine that the balance between these massive derivatives is disturbed, giving significant extrapolation errors very quickly. This behavior was seen when predicting forces and moments with the AVMM on unseen validation data and is a well known problem \cite{ittc_maneuvering_2008}.

\sphinxAtStartPar
The MAVMM has fewer hydrodynamic derivatives with lower multicollinearity and minor extrapolation errors. Including propeller thrust in the manoeuvring model made it possible to obtain high accuracy with fewer hydrodynamic derivatives. Another problem with a too complex model is that the standard manoeuvres used in this paper does not follow the aspect of persistence of excitation, so that some of the hydrodynamic derivatives might not be identifiable \cite{revestido_herrero_two-step_2012}. During zigzag tests, the model is for instance exposed to only two rudder angles for a majority of the data. A series of step responses as used in \cite{miller_ship_2021} gives a better excitation, but requires a lot of space, which is possible at lake experiments, but not in a narrow basin.

\sphinxAtStartPar
The close integration with the EKF makes this method very convenient to use in online applications. The hydrodynamic derivatives are however not updated online. The regression needs to be rerun, which is a rapid procedure with the OLS regression.  The entire time series history or only more recent parts, can be used depending on how much the model should remembered.


\section{Conclusions}
\label{\detokenize{10.01_conclusions:conclusions}}\label{\detokenize{10.01_conclusions:id1}}\label{\detokenize{10.01_conclusions::doc}}
\sphinxAtStartPar
This paper presented a new method for system identification of ship manoeuvring dynamics using a new parameter estimation method  applied to manoeuvring models. The proposed method includes:
\begin{itemize}
\item {} 
\sphinxAtStartPar
A model development process for robust models with good generalization, where the validation set should have larger yaw rates, drift angles and rudder angles compared to the training set.

\item {} 
\sphinxAtStartPar
A new parameter estimation method which includes:
\begin{itemize}
\item {} 
\sphinxAtStartPar
Preprocess measurement data with EKF + RTS run in iteration with initial guess from semi\sphinxhyphen{}empirical formulas.

\item {} 
\sphinxAtStartPar
Inverse dynamics regression

\end{itemize}

\end{itemize}

\sphinxAtStartPar
It was shown that:
\begin{itemize}
\item {} 
\sphinxAtStartPar
The new method can predict Turning circles with less than 5 \% error in advance and tactical diameter for the wPCC and KVLCC2 test cases, which should be considered sufficient considering the margin to the corresponding limits in the IMO standard for both ships.

\item {} 
\sphinxAtStartPar
For the KVLCC2 case with the manoeuvring model trained on zigzag model test data from the towing tank at HSVA, it was possible to reproduce the turning circle model test data from MARIN with reasonable accuracy. This example is one exciting application where the new method can be used to extend the model test from a narrow towing tank to predict turning circles.

\item {} 
\sphinxAtStartPar
The inverse dynamics regression had higher accuracy when the proposed preprocessor was used instead of low\sphinxhyphen{}pass filters.

\end{itemize}

\sphinxAtStartPar
Finally, it is concluded that the proposed method can potentially improve the system identification of ship manoeuvring dynamics. The KVLCC2 test case results with the new method are closer to the model tests compared to a similar study \cite{he_nonparametric_2022}. Adding the prior knowledge from semi\sphinxhyphen{}empirical formulas as the initial guess into the EKF iteration, adding the thrust model and adopting the complexity of the manoeuvring model by reducing the number of hydrodynamic derivatives are all contributing to the improved performance.


\section{Acknowledgements}
\label{\detokenize{acknowledgements:acknowledgements}}\label{\detokenize{acknowledgements::doc}}
\sphinxAtStartPar
The authors would like to acknowledge the Energimyndigheten (Swedish Energy Agency) project: Data\sphinxhyphen{}driven energy efficiency of ships (project: 49301\sphinxhyphen{}1) for providing the resources to prepare this paper. They would also thank all personnel at SSPA who have been involved in creating the model test results, building the ship models, and conducting the experiments.
\phantomsection\label{\detokenize{bibligraphy:id1}}







\renewcommand{\indexname}{Index}

\bibliography{references}
\end{document}